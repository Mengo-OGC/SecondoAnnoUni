\section{Statistica Descrittiva}

\begin{quotation}
	"Si occupa di presentare/descrivere i dati raccoli in un indiagine nel modo migliore possibile (sintetico, comunicativo)."
\end{quotation}

\definizione{Popolazione}{Insieme su cui vogliamo effettuare un'indagine.}{d:popolazione}

\definizione{Carattere}{Un carattere è quello che vogliamo studiare dalla popolazione.}{d:carattere}

\begin{itemize}
	\item \textbf{Carattere Quantitativo} se assumiamo valori numerici che esprimono una misura.
	\item \textbf{Carattere Qualitativo} se non è qualitativo.
\end{itemize}

I caratteri sarebbero distinguibili anche in \textbf{discreti e continui}.

\definizione{Unità (Statistica)}{Un unità (statistica) è un elemento della popolazione.}{d:uniStatistica}

\definizione{Campione}{Un campione è un sotto insieme (rappresentativo) della popolazione del quale possiamo determinare il valore del campione.}{d:campione}

\definizione{Modalità}{La modalità sono i valori che può assumere il carattere.}{d:modalità}

\definizione{Classi}{Se le modalità sono molto numerose (o infinite) è conveniente raggrupparle in \textbf{classi}.

Una classe, solitamente, è determinata dal \textbf{confine inferiore e superiore}.
Il \textbf{valore centrale} di una classe è la media dei confini.

Se abbiamo un solo confine chi fa l'indagine può decidere un valore rappresentativo come valore centrale.}{d:classi}

\definizione{Frequenza (assoluta)}{La frequenza (assoluta) di una modalità è il numero di volte in cui compare nel campione.}{d.freqAss}

\definizione{Frequenza relativa}{La frequenza relativa è il rapporto fra la frequenza assoluta e la cardinalità del campione. \[F_R=\frac{F_A}{|campione|}\]}{d:freqRel}

\definizione{Istogramma}{L'istogramma è un grafico che rappresenta i risultati di un'indagine.

Ad'ogni modalità (classe) è associato un rettangolo con base proporzionale all'ampiezza e area proporzionale alla frequenza.}{d:istogramma}

\begin{center}
	\begin{tikzpicture}
		\draw[draw=black, -latex, thin, solid] (-5.00,0.00) -- (5.00,0.00);
		\node[black, anchor=south west] at (-5.06,-0.75) {0};
		\node[black, anchor=south west] at (-4.06,-0.75) {10};
		\node[black, anchor=south west] at (-3.06,-0.75) {20};
		\node[black, anchor=south west] at (-2.06,-0.75) {30};
		\node[black, anchor=south west] at (-1.06,-0.75) {40};
		\node[black, anchor=south west] at (-0.06,-0.75) {50};
		\node[black, anchor=south west] at (0.94,-0.75) {60};
		\node[black, anchor=south west] at (1.94,-0.75) {70};
		\node[black, anchor=south west] at (2.94,-0.75) {80};
		\node[black, anchor=south west] at (3.94,-0.75) {90};
		\draw[draw=black, thin, solid] (-5.00,0.00) circle (0.1);
		\draw[draw=black, thin, solid] (-4.00,0.00) circle (0.1);
		\draw[draw=black, thin, solid] (-3.00,0.00) circle (0.1);
		\draw[draw=black, thin, solid] (-2.00,0.00) circle (0.1);
		\draw[draw=black, thin, solid] (-1.00,0.00) circle (0.1);
		\draw[draw=black, thin, solid] (0.00,0.00) circle (0.1);
		\draw[draw=black, thin, solid] (1.00,0.00) circle (0.1);
		\draw[draw=black, thin, solid] (2.00,0.00) circle (0.1);
		\draw[draw=black, thin, solid] (3.00,0.00) circle (0.1);
		\draw[draw=black, thin, solid] (4.00,0.00) circle (0.1);
		\draw[draw=black, thin, solid] (-5.00,0.00) -- (-5.00,2.00);
		\draw[draw=black, thin, solid] (-5.00,2.00) -- (-3.00,2.00);
		\draw[draw=black, thin, solid] (-3.00,2.00) -- (-3.00,0.00);
		\draw[draw=black, thin, solid] (-3.00,1.50) -- (-1.00,1.50);
		\draw[draw=black, thin, solid] (-1.00,1.50) -- (-1.00,0.00);
		\draw[draw=black, thin, solid] (-1.00,0.50) -- (1.00,0.50);
		\draw[draw=black, thin, solid] (1.00,0.50) -- (1.00,0.00);
		\draw[draw=black, thin, solid] (1.00,0.50) -- (1.00,3.00);
		\draw[draw=black, thin, solid] (1.00,3.00) -- (2.00,3.00);
		\draw[draw=black, thin, solid] (2.00,3.00) -- (2.00,0.00);
		\draw[draw=black, thin, solid] (2.00,3.00) -- (2.00,3.50);
		\draw[draw=black, thin, solid] (2.00,3.50) -- (2.50,3.50);
		\draw[draw=black, thin, solid] (2.50,3.50) -- (2.50,0.00);
		\draw[draw=black, thin, solid] (2.50,1.50) -- (4.00,1.50);
		\draw[draw=black, thin, solid] (4.00,1.50) -- (4.00,0.00);
	\end{tikzpicture}
\end{center}

\definizione{Media Campionaria}{La media campionaria si effettua su un carattere quantitativo su un campione di $n$ elementi. \[\overline{x}=\overline{x}_n=\frac{1}{n}(x_1+\dots+x_n)\]}{d:medCamp}

Ricordiamo delle proprietà delle sommatorie:
\begin{itemize}
	\item $\sum_{i=1}^n(a_i+b_i)=\sum_{i=1}^na_i+\sum_{i=1}^nb_i$
	\item $\sum_{i=1}^nc=nc$
	\item $\sum_{i=1}^n(ca_i)=c\sum_{i=1}^na_i$
	\item $\sum_{i=1}^n(\sum_{j=1}^n(a_ib_j))=(\sum_{i=1}^na_i)\cdot(\sum_{i=1}^nb_i)$
\end{itemize}

\definizione{Linearità della media}{Siano $x,y,z$ tre caratteri legati dalla relazione $z=ax+by$ con costanti $a,b$. \[\overline{z}=a\overline{x}+b\overline{y}\]}{d:linMed}

Caso particolare è con $y=1$ ottenendo così $z=ax+b$ e quindi $\overline{z}=a\overline{x}+b$. Può essere usato per semplificare i conti.

\dimostrazione{d:linMed}{\[\overline{z}=\frac{1}{n}\sum_{i=1}^nz_i=\frac{1}{n}\sum_{i=1}^n(ax_i+by_i)=\frac{1}{n}a\sum_{i=1}^nx_i+\frac{1}{n}b\sum_{i=1}^ny_i=a\overline{x}+b\overline{y}\]}

