\section{Statistica Descrittiva}

\begin{quotation}
	"Si occupa di presentare/descrivere i dati raccoli in un indiagine nel modo migliore possibile (sintetico, comunicativo)."
\end{quotation}

\definizione{Popolazione}{Insieme su cui vogliamo effettuare un'indagine.}{d:popolazione}

\definizione{Carattere}{Un carattere è quello che vogliamo studiare dalla popolazione.}{d:carattere}

\begin{itemize}
	\item \textbf{Carattere Quantitativo} se assumiamo valori numerici che esprimono una misura.
	\item \textbf{Carattere Qualitativo} se non è qualitativo.
\end{itemize}

I caratteri sarebbero distinguibili anche in \textbf{discreti e continui}.

\definizione{Unità (Statistica)}{Un unità (statistica) è un elemento della popolazione.}{d:uniStatistica}

\definizione{Campione}{Un campione è un sotto insieme (rappresentativo) della popolazione del quale possiamo determinare il valore del campione.}{d:campione}

\definizione{Modalità}{La modalità sono i valori che può assumere il carattere.}{d:modalità}

\definizione{Classi}{Se le modalità sono molto numerose (o infinite) è conveniente raggrupparle in \textbf{classi}.

Una classe, solitamente, è determinata dal \textbf{confine inferiore e superiore}.
Il \textbf{valore centrale} di una classe è la media dei confini.

Se abbiamo un solo confine chi fa l'indagine può decidere un valore rappresentativo come valore centrale.}{d:classi}

\definizione{Frequenza (assoluta)}{La frequenza (assoluta) di una modalità è il numero di volte in cui compare nel campione.}{d.freqAss}

\definizione{Frequenza relativa}{La frequenza relativa è il rapporto fra la frequenza assoluta e la cardinalità del campione. \[F_R=\frac{F_A}{|campione|}\]}{d:freqRel}

\definizione{Istogramma}{L'istogramma è un grafico che rappresenta i risultati di un'indagine.

Ad'ogni modalità (classe) è associato un rettangolo con base proporzionale all'ampiezza e area proporzionale alla frequenza.}{d:istogramma}

\begin{center}
	\begin{tikzpicture}
		\draw[draw=black, -latex, thin, solid] (-5.00,0.00) -- (5.00,0.00);
		\node[black, anchor=south west] at (-5.06,-0.75) {0};
		\node[black, anchor=south west] at (-4.06,-0.75) {10};
		\node[black, anchor=south west] at (-3.06,-0.75) {20};
		\node[black, anchor=south west] at (-2.06,-0.75) {30};
		\node[black, anchor=south west] at (-1.06,-0.75) {40};
		\node[black, anchor=south west] at (-0.06,-0.75) {50};
		\node[black, anchor=south west] at (0.94,-0.75) {60};
		\node[black, anchor=south west] at (1.94,-0.75) {70};
		\node[black, anchor=south west] at (2.94,-0.75) {80};
		\node[black, anchor=south west] at (3.94,-0.75) {90};
		\draw[draw=black, thin, solid] (-5.00,0.00) circle (0.1);
		\draw[draw=black, thin, solid] (-4.00,0.00) circle (0.1);
		\draw[draw=black, thin, solid] (-3.00,0.00) circle (0.1);
		\draw[draw=black, thin, solid] (-2.00,0.00) circle (0.1);
		\draw[draw=black, thin, solid] (-1.00,0.00) circle (0.1);
		\draw[draw=black, thin, solid] (0.00,0.00) circle (0.1);
		\draw[draw=black, thin, solid] (1.00,0.00) circle (0.1);
		\draw[draw=black, thin, solid] (2.00,0.00) circle (0.1);
		\draw[draw=black, thin, solid] (3.00,0.00) circle (0.1);
		\draw[draw=black, thin, solid] (4.00,0.00) circle (0.1);
		\draw[draw=black, thin, solid] (-5.00,0.00) -- (-5.00,2.00);
		\draw[draw=black, thin, solid] (-5.00,2.00) -- (-3.00,2.00);
		\draw[draw=black, thin, solid] (-3.00,2.00) -- (-3.00,0.00);
		\draw[draw=black, thin, solid] (-3.00,1.50) -- (-1.00,1.50);
		\draw[draw=black, thin, solid] (-1.00,1.50) -- (-1.00,0.00);
		\draw[draw=black, thin, solid] (-1.00,0.50) -- (1.00,0.50);
		\draw[draw=black, thin, solid] (1.00,0.50) -- (1.00,0.00);
		\draw[draw=black, thin, solid] (1.00,0.50) -- (1.00,3.00);
		\draw[draw=black, thin, solid] (1.00,3.00) -- (2.00,3.00);
		\draw[draw=black, thin, solid] (2.00,3.00) -- (2.00,0.00);
		\draw[draw=black, thin, solid] (2.00,3.00) -- (2.00,3.50);
		\draw[draw=black, thin, solid] (2.00,3.50) -- (2.50,3.50);
		\draw[draw=black, thin, solid] (2.50,3.50) -- (2.50,0.00);
		\draw[draw=black, thin, solid] (2.50,1.50) -- (4.00,1.50);
		\draw[draw=black, thin, solid] (4.00,1.50) -- (4.00,0.00);
	\end{tikzpicture}
\end{center}

\definizione{Media Campionaria}{La media campionaria si effettua su un carattere quantitativo su un campione di $n$ elementi. \[\overline{x}=\overline{x}_n=\frac{1}{n}(x_1+\dots+x_n)\]}{d:medCamp}

Ricordiamo delle proprietà delle sommatorie:
\begin{itemize}
	\item $\sum_{i=1}^n(a_i+b_i)=\sum_{i=1}^na_i+\sum_{i=1}^nb_i$
	\item $\sum_{i=1}^nc=nc$
	\item $\sum_{i=1}^n(ca_i)=c\sum_{i=1}^na_i$
	\item $\sum_{i=1}^n(\sum_{j=1}^n(a_ib_j))=(\sum_{i=1}^na_i)\cdot(\sum_{i=1}^nb_i)$
\end{itemize}

\definizione{Linearità della media}{Siano $x,y,z$ tre caratteri legati dalla relazione $z=ax+by$ con costanti $a,b$. \[\overline{z}=a\overline{x}+b\overline{y}\]}{d:linMed}

Caso particolare è con $y=1$ ottenendo così $z=ax+b$ e quindi $\overline{z}=a\overline{x}+b$. Può essere usato per semplificare i conti.

\dimostrazione{d:linMed}{\[\overline{z}=\frac{1}{n}\sum_{i=1}^nz_i=\frac{1}{n}\sum_{i=1}^n(ax_i+by_i)=\frac{1}{n}a\sum_{i=1}^nx_i+\frac{1}{n}b\sum_{i=1}^ny_i=a\overline{x}+b\overline{y}\]}

\definizione{Media Ponderata}{La media ponderata si usa per dare diversa importanza (peso) ai vari elementi d'un campione. \[\overline{x_w}=\frac{x_1w_1+\dots+x_nw_n}{w_1+\dots+w_n}\]}{d:medPond}

\osservazione{La media campionaria si ottiene dando peso 1 a tutti gli elementi.}

\definizione{Varianza}{La varianza di $x$ è la media dei quadrati della distanza da $\overline{x}$. \[\sigma_x^2=\frac{1}{n}\sum_{i=1}^n(x_i-\overline{x})^2\]}{d:varianza}
In altre parole indica la dispersione dei valori, ovvero da una misura di qunato i dati sono lontanti tra dolo.

\dimostrazione{d:varianza}{Ricordiamo la formula della parabola $at^2+bt+c=0, a>0$ e con vertice d'ascissa $-b/2a$.

Possiamo fissare un punto a caso $t$. Consideriamo la media dei quadrati delle distanze date come: \[\frac{(x_1-t)^2+\dots+(x_n-t)^2}{n}\] Per quale $t$ questa quantità è minima? $\frac{1}{n}\sum_{i=1}^n(x_i-t)^2=\frac{1}{n}(\sum_{i=1}^n(t^2-2x_1t+x_1^2))=t^2-\frac{2t}{n}\sum_{i=1}^nx_i+\frac{1}{n}\sum_{i=1}^nx_i^2=t^2-2\overline{x}t+\frac{1}{n}\sum_{i=1}^nx_i^2$.
Il valore minimo lo otteniamo per $t=\frac{-(-2\overline{x})}{2 \cdot 1}=\overline{x}$.}

\osservazione{Dal calcolo precedente con $\overline{x}, \sigma_x^2=\overline{x}^2-2\overline{x}\overline{x}-\overline{x^2}=\overline{x^2}-\overline{x}^2$.}

\osservazione{La varianza è sempre positiva.}

\proprieta{1° varianza}{La varianza rimane invariata per traslazioni di costanti. \[\sigma_y^2=\sigma_x^2\]}{prop:var1}

\dimostrazione{prop:var1}{Sia $y=x+c$ con $c$ costante. Otteniamo $\overline{y}=\overline{x}+c$ ovvero $\overline{y^2}=\overline{x^2+2cx+c^2}=\overline{x^2}+2c\overline{x}+c^2$ che ci da $\sigma_y^2=\overline{y^2}-\overline{y}^2=\overline{x^2}+2c\overline{x}+c^2-(\overline{x}^2+2c\overline{x}+c^2)=\sigma_x^2$.}

\proprieta{2° varianza}{La varianza aumenta per prodotto con costanti. \[\sigma_y^2=a^2\sigma_x^2\]}{prop:var2}

\dimostrazione{prop:var2}{Sia $y=ax$ con $a$ costante. Otteniamo $\overline{y}=a\overline{x}$ ovvero $\overline{y^2}=\overline{a^2x^2}=a^2\overline{x^2}$ che ci da $\sigma_y^2=\overline{y^2}-\overline{y}^2=a^2\overline{x^2}-a^2\overline{x}^2=a^2\sigma_x^2$.}

\definizione{Correlazione Positiva}{Dati due caratteri $x$ e $y$ diciamo che sono positivamente correlati se al crescere di uno ci aspettiamo che cresca anche l'altro.}{d:posCor}

\definizione{Correlazione Negativa}{Dati due caratteri $x$ e $y$ diciamo che sono negativamente correlati se al crescere di uno ci aspettiamo che l'altro diminuisca.}{d:negCor}

\definizione{Covarianza}{La covarianza di $x$ e $y$ è \[\sigma_{x,y}=\frac{1}{n}\sum_{i=1}^n(x_i-\overline{x})(y_i-\overline{y})=\overline{xy}-\overline{x}\overline{y}\]}{d:covarianza}
In altre parole se $x$ e $y$ sono positivamente correlate allora $(x_i-\overline{x})$ e $(y_i-\overline{y})$. Ci aspettiamo che siano concordi e quindi che il prodotto sia maggiore di zero.

\osservazione{$\sigma_{x,x}=\sigma_x^2$.}

\dimostrazione{d:covarianza}{$\sigma_{x,y}=\frac{1}{n}\sum_{i=1}^n(x_i-\overline{x})(y_i-\overline{y})=\frac{1}{n}\sum_{i=1}^n(x_iy_i)-x_i\overline{y}-\overline{x}y_i+\overline{xy}=\overline{xy}-\overline{x}\overline{y}-\overline{x}\overline{y}+\overline{x}\overline{y}=\overline{xy}-\overline{x}\overline{y}$.}

\definizione{Retta ai minimi quadrati}{La retta ai minimi quadratici è la retta che minizza i quadrati degli errori (cioè delle lunghezze dei segmenti verticali.)}{def:rettaMinQuad}

La somma dei quadrati degli errori è \[S(m,q)=\sum_{i=1}^n(y_i-mx_i-q)^2\] vogliamo $m$ e $q$ in modo che queta quantità sia minima. \[S(m,q)=\sigma_y^2+m^2\sigma_x^2-em\sigma_{x,y}+(\overline{y}-q-m\overline{x})^2\]

\begin{center}
	\begin{tikzpicture}
		\draw[draw=black, -latex, thin, solid] (-2.00,-2.00) -- (4.00,-2.00);
		\draw[draw=black, -latex, thin, solid] (-2.00,-2.00) -- (-2.00,2.00);
		\node[black, anchor=south west] at (3.94,-2.75) {x};
		\node[black, anchor=south west] at (-3.06,1.25) {y};
		\node[black, anchor=south west] at (-3.06,-2.75) {0};
		\draw[draw=black, thin, solid] (-1.00,-1.50) circle (0.1);
		\draw[draw=black, thin, solid] (-1.50,-1.00) circle (0.1);
		\draw[draw=black, thin, solid] (-1.00,-0.50) circle (0.1);
		\draw[draw=black, thin, solid] (0.00,-0.50) circle (0.1);
		\draw[draw=black, thin, solid] (0.00,0.50) circle (0.1);
		\draw[draw=black, thin, solid] (1.00,-0.50) circle (0.1);
		\draw[draw=black, thin, solid] (1.00,1.50) circle (0.1);
		\draw[draw=black, thin, solid] (2.00,0.50) circle (0.1);
		\draw[draw=black, thin, solid] (2.00,2.00) circle (0.1);
		\draw[draw=black, thin, solid] (2.50,1.00) circle (0.1);
		\draw[draw=black, thin, solid] (-2.00,-2.00) -- (2.50,2.00);
		\draw[draw=black, ultra thin, solid] (-1.50,-1.00) -- (-1.50,-1.50);
		\draw[draw=black, ultra thin, solid] (-1.00,-1.50) -- (-1.00,-0.50);
		\draw[draw=black, ultra thin, solid] (0.00,-0.50) -- (0.00,0.50);
		\draw[draw=black, ultra thin, solid] (1.00,-0.50) -- (1.00,1.50);
		\draw[draw=black, ultra thin, solid] (2.00,0.50) -- (2.00,2.00);
		\draw[draw=black, ultra thin, solid] (2.50,1.00) -- (2.50,2.00);
	\end{tikzpicture}
\end{center}

\osservazione{Sia $\overline{y}=m\overline{x}+q$, il punto $(\overline{x}, \overline{y}) \in $ retta è $S(m,q)=\sigma_y^2+m^2\sigma_x^2-2m\sigma_{x,y}\implies m=\frac{-b}{2a}=-\sigma_{x,y}\frac{1}{2\sigma_x^2}=\frac{\sigma_{x,y}}{\sigma_x^2}$.}