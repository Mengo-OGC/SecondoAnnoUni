\section{Probabilità Base}

\begin{quotation}
	"Si occupa di prevedere quanto è facile/possibile che qualcosa accada. Consiste in passaggi logici rigorosi partendo da un modello fisso (spazio di probabilità)."
\end{quotation}

\definizione{Fenomeno}{Un fenomeno è qualcosa che accade e che porta ad'un esito o risultato.}{d:fenomeno}
Un fenomeno può essere:
\begin{itemize}
	\item \textbf{Deterministico} se il risultato può essere predetto con esattezza.
	\item \textbf{Aleatorio} se il risultato è imprevedibile.
\end{itemize}

\definizione{Evento}{Un evento è un insieme di possibili risultati.}{d:evento}

\definizione{Valutazione di probabilità}{La valutazione di probabilità è una funzione che ad'ogni evento associa un numero tanto più grande quanto riteniamo che l'evento possa accadere.}{d:valProb}

\definizione{Evento coerente}{Sia $\Omega$ l'insieme dei risultati. Un evento è un sotto insieme di $\Omega$ di cui ha senso calcolare la valutazione di probabilità, ossia:
\begin{enumerate}
	\item Se $A$ e $B$ sono eventi allora $A\cap B$ è un evento.
	\item Se $A_1,\dots$ sono eventi allora $\bigcup_i A_i$ è un evento.
	\item Se $A$ è un evento allora $A^C$ è un evento.
	\item $\Omega$ è un evento.
\end{enumerate}}{d:eventoProb}
\osservazione{Una collezione di sotto insiemi di $\Omega$ che soddisfa tutti i presupposti si dice \textbf{famiglia coerente d'eventi}, con simbolo $\mathscr{A}$, ($\mathscr{A} \in \Omega$).}

\medskip

\noindent La valutazione di probabilità, $P:\mathscr{A}\to\mathbb{R}_{\geqslant0}$, deve soddisfare le proprietà:
\begin{enumerate}
	\item $P(\Omega)=1$.
	\item $P(A)\geqslant0,\forall \ evento \ A$.
	\item Se $A_1,\dots$ sono eventi disgiunti allora \[P(\bigcup_{i=1}^\infty A_i)=\sum_{i=1}^\infty P(A_i)\]
\end{enumerate}
\osservazione{$P$ non ha dominio $\Omega$ ma l'insieme degli eventi. Non calcoliamo la $P$ di un risultato ma di un evento.}

\osservazione{Non tutti i sotto insiemi di $\Omega$ sono eventi.}

\definizione{Probabilità uniforme}{
La probabilità è uniforme:
\begin{enumerate}
	\item Se $\Omega$ è finito con $|\Omega|=n$.
	\item Se ogni sotto insieme di $\Omega$ è un evento.
	\item Se $\omega_1$ e $\omega_2$ sono due risultati allora $P(\{\omega_1\})=P(\{\omega_2\})$.
\end{enumerate}
Dalle proprietà della valutazione di probabilità deduciamo anche \[P(\omega_1)+\dots+P(\omega_n)=P(\Omega)=1\]
In generale abbiamo quindi $P(\omega)=\frac{1}{n}, \forall \omega \in \Omega$.}{d:probUnif}

\definizione{Probabilità eventi}{Sia $A \subseteq \Omega, A=\{\omega_1,\dots,\omega_k\}$ \[P(A)=\frac{|A|}{|\Omega|}=\frac{\#risultati \ favorevoli}{\#risultati \ possibili}\]}{d:probEv}
Ragionamento: Se $|A|=k$, allora
\begin{align*}
	P(A)&=P(\omega_1)+\dots+P(\omega_k)\\
	&=\frac{1}{n}+\dots+\frac{1}{n}=\frac{k}{n}
\end{align*}
\osservazione{Vale uno spazio con proprietà uniforme.}

\definizione{Non-esempio}{Un non-esempio è qualcosa che non funziona.}{d:nonEsempio}

\mysubsection{Considerazioni elementari}
\begin{itemize}
	\item Se $E$ è un evento allora $E^C$ è un evento.
	\item $E\cup E^C=\Omega$ è un uninone disguinta.
	\item Se $P(E)+P(E^C)=1$ allora $P(E^C)=1-P(E)$.
\end{itemize}
\osservazione{Il principio di inclusione-esclusione vale anche per l'unione di più di due probabilità.}

\definizione{Probabilità condizionata}{Chiamiamo probabilità condizionata la probabilità che accada l'evento $B$ sapendo che accade l'evento $A$ prima di $B$.\[P(B\mid A)=\frac{P(A\cap B)}{P(A)}\]}{d:probCond}
In altre parole riparametrizziamo la probabilità di $B$ da $\Omega$ al sotto insieme codiviso con $A$.

\osservazione{Ponendo $P'(B)=P(B\mid A)$ allora $P'$ soddisfa le proprietà delle valutazioni di probabilità:
\begin{itemize}
	\item $P'(\Omega)=P(\Omega\mid A)=\frac{P(\Omega \cap A)}{P(A)}=1$.
	\item $P'(A_1\cup A_2)=P'(A_1)+P'(A_2)$ se $A_1,A_2$ sono disgiunti.
\end{itemize}}
\osservazione{$P(A\cap B)=P(B\mid A)P(A)$.}

\definizione{Formula delle probabilità totali}{Sia $\Omega$ partizionato in $\{A_1,A_2,\dots\}$. Per sapere la probabilità d'un evento $B$ condizionato da un qualsiasi altro evento. \[P(B)=\sum_{i=1}^nP(B\cap A_i)=\sum_{i=1}^nP(B\mid A_i)P(A_i)\]}{formulaPorblTot}

\definizione{Formula di Bayes}{Siano $A,B$ eventi, allora la formula di Bayes  \[P(B\mid A)=\frac{P(A\mid B)P(B)}{P(A)}\]inverte le probabilità condizionate, $P(B\mid A)\leftrightarrow  P(A\mid B)$.}{d:fBaeys}
\dimostrazione{d:fBaeys}{Siano $P(A\mid B)=\frac{P(A\cap B)}{P(B)}$ e $P(B\mid A)=\frac{P(A\cap B)}{P(A)}$. Allora $P(A\cap B)=P(A\mid B)P(B)=P(B\mid A)P(A)=P(A  \cap B)$.}

\definizione{Eventi indipendenti}{$A$ e $B$ sono eventi indipendenti se sapere che accade uno dei due non cambia la probabilità che accada l'altro. \[P(B\mid A)=P(B)\] \[P(A\mid B)=P(A)\]}{d:evDip}
\osservazione{$P(A\cap B)=P(A)P(B)$ dato che $P(A\mid B)=\frac{P(A\cap B)}{P(B)}=P(A)$.}