\section{Statistica Inferenziale}

\begin{quotation}
	"Scopo di dedurre proprietà di una popolazione dalla conoscienza dei dati di un campione."
\end{quotation}

In generale in un indagine statistica non conosciamo $\mu$ o $\sigma$ e quindi il nostro obbiettivo è di stimare, in particolare $\mu$.

\begin{align*}
	P(-Z_c\frac{\sigma}{\sqrt{n}}<&\overline{X_n}-\mu<Z_c\frac{\sigma}{\sqrt{n}})=c\\
	P(-\overline{X_n}-Z_c\frac{\sigma}{\sqrt{n}}<&-\mu<-\overline{X_n}+Z_c\frac{\sigma}{\sqrt{n}})=c\\
	P(\overline{X_n}-Z_c\frac{\sigma}{\sqrt{n}}<&-\mu<\overline{X_n}Z_c\frac{\sigma}{\sqrt{n}})=c
\end{align*}

$\overline{X_n}$ è la media ottenuta sul campione e la conosciamo. $\sigma$ è la varianza della popolazione ee non la conosciamo.

Sostituendo $\sigma^2$ con $\overline{\sigma_n}$, la varianza del campione, otteniamo \[\mathrm{Conf}(\overline{X_n}-Z_c\frac{\overline{\sigma_n}}{\sqrt{n}}<\mu<\overline{X_n}+Z_c\frac{\overline{\sigma_n}}{\sqrt{n}})=c\]

\subsection{Significatività d'un test}
Un test vuole stabilire se un prodotto è "buono" mediante una certa proprietà.
\begin{itemize}
	\item Decido il criterio di valutazione.
	\item Definisco $n$ il numero di tentativi da eseguire.
	\item Effettuo il test.
\end{itemize}
Il test è positivo se rientra nell'intervallo stimato, o negativo al'trimenti, con una certa significatività (possibilità di scartare un buono sbagliando).

