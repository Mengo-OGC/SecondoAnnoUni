\usepackage[italian]{babel} % imposta la lingua in italiano.
\usepackage{thmbox}			% ridifinizione stile blocco definizioni, teoremi, ecc tranne dimostrazione.
\usepackage{tabularx}		% fa tabelle con colonne ridimensionabili.
\usepackage{amsmath}    	% formule potenziate, allineamento, casi, ecc.
\usepackage{amssymb}		% simboli aggiuntivi (insiemi, operatori logici, ecc.).
\usepackage{amsfonts}	    % font matematici aggiuntivi (es. \mathbb).
\usepackage[table]{xcolor}  % permette di colorare righe/colonne.
\usepackage{fancyhdr}		% stile impaginazione.
\usepackage{tikz}			% per fare grafici

% link interni alle pagine e indice auto linkato.
\usepackage[colorlinks=true, linkcolor=black]{hyperref}

% impaginazione personalizzata:
\pagestyle{fancy}
\fancyhf{}
\fancyhead[L]{\hyperref[toc]{$\leftarrow$Indice}}	% link a indice sinistra alto
\fancyhead[R]{\leftmark}							% nome sezione destra alto
\fancyfoot[C]{\thepage}								% numero pagina centrale basso

% ambienti personalizzati
\newtheorem[L]{teo}{Teorema}[section]
\newtheorem[L]{deff}{Definizione}[section]
\newtheorem[S]{propo}{Proposizione}[section]
\newtheorem[S]{propr}{Proprietà}[section]
\newtheorem[S]{cor}{Corollario}[section]

% permettono nelle dimostrazioni di avere "NomeAmbiente Numero"
\providecommand*{\teoautorefname}{Teorema}
\providecommand*{\deffautorefname}{Definizione}
\providecommand*{\propoautorefname}{Proposizione}
\providecommand*{\proprautorefname}{Proprietà}
\providecommand*{\corautorefname}{Corollario}

% definizione ambiente generico
\newcommand{\bloccoTeorico}[6]{
	\begin{#1}[#2]
		\label{#4}%
		#3
	\end{#1}
	\addcontentsline{toc}{subsection}{#5~#6~(#2)}
}

% definizione ambienti specifici. Nota: 4° parametro testo per indice
\newcommand{\teorema}[3]{
	\bloccoTeorico{teo}{#1}{#2}{#3}{Teorema}{\theteo}
}
\newcommand{\definizione}[3]{
	\bloccoTeorico{deff}{#1}{#2}{#3}{Definizione}{\thedeff}
}
\newcommand{\proposizione}[3]{
	\bloccoTeorico{propo}{#1}{#2}{#3}{Proposizione}{\thepropo}
}
\newcommand{\proprieta}[3]{
	\bloccoTeorico{propr}{#1}{#2}{#3}{Proprietà}{\thepropr}
}
\newcommand{\corollario}[3]{
	\bloccoTeorico{cor}{#1}{#2}{#3}{Corollario}{\thecor}
}

% ambiente dimostrazione che mostra nome e numero cliccabile della cosa da dimostrare
\newcommand{\dimostrazione}[2]{
	\begin{proof}[\autoref{#1}] #2 \end{proof}
}

% comandi semplici fomattazione
\newcommand{\osservazione}[1]{
	\textit{Osservazione:} #1
}

% tabella con colonne adattabili, non indentata e di dimensione del testo
\newcommand{\tabella}[3]{
	\noindent
	\begin{tabularx}{\textwidth}{#1}
		\hline
		\rowcolor{gray!10}
		#2 \\
		\hline
		#3
		\hline
	\end{tabularx}
}

\newcommand{\mysubsection}[1]{
	\subsection*{#1}
	\addcontentsline{toc}{subsection}{#1}
}
