\documentclass{article}
\usepackage{listings}
\usepackage{xcolor}

\renewcommand{\lstlistingname}{Codice}		% titoletto dei caption

\lstdefinestyle{mystyle}{
	basicstyle=\ttfamily\small,				% stile testo leggermente più piccolo e monospace
	commentstyle=\itshape\color{gray},		% commenti grigi in italico
	keywordstyle=\color{blue},				% parole chiave linguaggio in blu
	stringstyle=\color{green!40!black},		% stringhe in "verde"
	numbers=left,							% numeri riga visibili sulla sinistra
	numberstyle=\tiny\color{gray},			% numeri riga piccolissimi e grigi
	breaklines=true,						% spezza righe troppo lunghe su più righe
	showspaces=false,						% seguenti 3 comandi nascondono caratteri indesiderati
	showstringspaces=false,
	showtabs=false,
	tabsize=2,								% larghezza dei tab 2 spazi
	xleftmargin=\parindent					% allinea blocco ai paragrafi
}

\lstset{style=mystyle}						% applico stile definito

\newcommand{\multiline}[3]{\lstinputlisting[language=Java, caption={#2}, label={#1}]{#3}}
\newcommand{\inline}[1]{\noindent\lstinline[language=Java]@#1@}

\newcommand{\colTwo}[4]{
	\begin{minipage}[t]{0.#3\textwidth}
		#1
	\end{minipage}
	\hfill
	\begin{minipage}[t]{0.#4\textwidth}
		#2
	\end{minipage}\\[0.01\textheight]
}

\title{Programmazione ad'Oggetti}
\author{Leonardo Mengozzi}
\date{}
\begin{document}
	\maketitle
	\tableofcontents

	\section{Fasi sviluppo Sofware}
Un Programma (algoritmo) risolve una classe di problemi.

Un Sistema sofware fornisce varie funzionalità grazie alla cooperazione di componenti di diversa natura.

Fasi processo sviluppo:
\begin{enumerate}
	\item \textbf{Analisi}: che fare?. Definisce precisamente il problema, ingegnerizza i requisiti, modellizza il dominio.

	Sotto fase è interazioni col committente produce un contratto (aspetto e funzionalità).

	Un analisi è corretta se persone diverse giungono alla stessa soluzione.
	\item \textbf{Design}: come farlo? Definisce struttura software/soluzione.

	\begin{itemize}
		\item Design architetturale: Progettazione struttura complessiva dei moduli e macro relazioni.

		Come per OPP? Unico diagramma di 5-12 interfacce non isolate, accompagnato da prosa dei ruoli e relazioni.
		Seguendo Pattern MVC (thread vanno in C).
		\item Progetto di dettaglio: Progettazione più precisa forma moduli e relazioni più importanti.

		Ulteriori UML documentano progettazioni di dettaglio.
	\end{itemize}

	\item \textbf{Implementazione/codifica}: Quale algoritmo? Si realizza il sofware scegliendo le tecnologie adeguate.
	\item \textbf{Post-coficia}: Collaudo, Manutenzione, Deployment.

	Può impiegare fino 70\% se fasi precedenti fatte male/sbrigativamente (riscio Software crisis).
\end{enumerate}
Fasi 1-2 fatte dai senior e fase 3 junior (2-3 oggi unificate).
Tutte fasi fattibili dalla stessa persona.

Approcci di sviluppo:
\begin{itemize}
	\item \textbf{Cascata}: Fasi eseguite in ordine temporale.
	\item \textbf{Spirale}: Fasi svolte in ordine ciclicamente.
	\item \textbf{Fontata}: Si oscilla temporaneamente tra fase successiva/precedente.
	\item \textbf{Agile}.
\end{itemize}

Per piccoli progetti: Porre obbietivi intermedi incrementali (prototipo, avanzato, definitivo), con approccio a cascata per ognuno.

\subsection{Problem space vs Solution space}
\begin{itemize}
	\item Problem space sono le entità/relazioni/processi del mondo reale che formano il problema.
	\item Solution space sono le entità/relazioni/processi nel mondo artificiale (espresse nel linguaggio di programmazione).
\end{itemize}


Per passare dal Problem space al Solution solution si esegue un "mapping" che più semplice è meglio ho fatto le \textbf{astrazioni}\footnote{Strumento che semplifica sistemi informatici ma anche del mondo reale evidenziando "la parte importante". Si possono fare più livelli di astrazione.}.

I linguaggi di programmazione attuano l'astrazione coi loro costrutti, più o meno performanti, che rendono il mapping più o meno facile.
I linguaggi moderni hanno un livello d'astrazione lontano dall'Hardware, i suoi problemi e la gestione della memoria.

\subsection{Programmazione ad oggetti (OOP)}
\textbf{Vantaggi:} poche astrazioni chiave, mapping ottimo e semplice, estensibilità e riutilizzo, librerie auto costruite, C-like, esecuzione efficente. \textbf{Critiche:} necessaria disciplina.

\bigskip

\colTwo{\textbf{Classe:} Descrizione comportamento e forma oggetti. Indica come comunicare con i suoi oggetti, con messaggi che modificano stato e comportamento.}{\textbf{Oggetto:} Entità (istanza di classe) manipolabile, con memoria, che comunicano tramite le loro operazioni descritte dalla classe di appartenenza.}{45}{45}

Oggetti della stessa classe hanno comportamento e forma indentica, sono detti simili.
Un oggetto non cambia mai classe, semmai si elimina e sene crea il sostituto.
Nota: L'approccio OOP è usato anche in UML.

\bigskip

\textbf{Com'è fatto un buon oggetto:}
Un oggetto ha un interfaccia\footnote{Insieme dei metodi definiti dall'interfaccia con cui l'oggetto riceve messaggi.}, deve fornire un servizio, deve nascondere le implementazioni (riutilizzabili) e l'intero oggetto deve essere riutilizzabile tramite ereditarietà. Precisazioni:
\begin{itemize}
	\item Un oggetto fornisce un \textbf{sotto-servizio} dell'intero programma\footnote{Set di oggetti che si comunicano cosa fare.} (principio decomposizione). Linne giuda: 1 oggetto senza servizio si elimina, 2 oggetto con più servizi si divide.

	\item L'implementazione di un oggetto (logiche interne) devono essere note solo al creatore della classe (Information hidding), cosi facendo l'utilizzatore è tutelato, da modifiche interne, avendo una piccola visione del tutto. \textit{less is more}.

	\item Il creatore e l'utilizzatore riutilizzano le classi con gli approcci:
	\begin{enumerate}
		\item \textbf{has-a} (composizione), classe costituita da altre classi (oggetto ha come campi altri oggetti). Approccio dinamico, occultabile.
		\item \textbf{is-a} (ereditarietà), classe estende servizi di un'altra classe (oggetto ha campi/metodi di altri oggetti).
	\end{enumerate}
\end{itemize}

\section{Perchè Java?}
\begin{itemize}
	\item \textbf{Write once run everywhere}, eseguibile uvunque senza ricompilazione grazie JVM (HardWare virtuale, a stack) che processa un codice specifico "byte code", creando il corrispettivo eseguibile per ogni pc/os. Meno prestante.

	\item \textbf{Keep it simple, stupid}, in teoria non in pratica.
\end{itemize}

\subsection{Elementi fondamentali}
\colTwo{\textbf{Espressione}: fornisce un risulato, riusabile ove si attende un valore.}{\textbf{Comando}: istruzione da terminare con ";", non componibile con altre}{45}{45}

Alcune parole chiave del linguaggio: \inline{for, while, do, switch, if, break, continue, return, var, ecc}.

\subsection{Differenze con C}
\begin{itemize}
	\item Condizioni di \inline{if, for, while, do} restituiscono \inline{boolean}.

	\item Nel \inline{for} è possibile dichiare le variabili contatore (visibili solo internamente).

	\inline{for (<tipo> <c1>,...; <condizoneBoolean>, <modificaC1>,...) \{...\}}

	\item Java da \inline{unreachable statement, variable may not have been initialised, missing return statemet,ecc} come errori, C solo come warning. Approccio più "rigido" ma anche più "corretto".
\end{itemize}

\subsection{Gestione Oggetti in memoria}
Alla creazione di un oggetto, la \inline{new} chiama il gestore della memoria, si calcola la dimensine della memoria da allocare, si alloca, si inizializza e si restituisce il riferimento alla variabile.

La memoria occupata da un oggetto, non si sa com'è organizzata, ma contiene i campi non statici e il riferimento alla classe con la tabella dei metodi. Gli elementi statici sono allocati in una sezione dedicata.

Il \textbf{garbage collector} (componente della JVM) dealloca automaticamente memoria heap non più utilizzata direttamente o indirettamente. Un'oggetto continua a esistere dopo la fine esecuzione dello scope di una variabile che gli fa riferimento.
	\section{Struttura Programma Java}
Un programma Java è composta da librerie di classi del JDK, Package\footnote{Contenitori, gerarcici tra loro, di una decina di classi di alto livello con scopo comune} e Moduli\footnote{Insieme di Package costituente un frammento di codice autonomo.}, librerie di esterne e un insieme di classi fondamentali, come la class main\footnote{Un main è il punto d'accesso di un programma.}.
\inline{public static void main(String[] args) \{...\}}

Per importare le classi di una libreria:
\begin{itemize}
	\item \inline{import java...;}, importa una singola classe.
	\item \inline{import java...*;}, importa l'intero Package.
	\item \inline{import java.lang.*;}, importazione di default.
\end{itemize}

Nota: Il nome completo di una classe dipende dal Package in cui si trova.

\subsection{Esecuzione Programma}
\begin{enumerate}
	\item Salvare la classe in un file \textbf{"NomeClasse.java"}.
	\item Compilare con \textit{javac NomeFileClasse.java}. Genererà il \textbf{bytecode NomeFileClasse.class} per la JVM.
	\item Esegiure con \textit{java NomeFileClasse}. La JVM cercherà il main da cui partire a eseguire.
\end{enumerate}

Lavorando con più file: si compila tutto con \textit{javac *.java} poi si esegue solo la classe main.





\section{(Almost) Everything is an object}
Le variabili, contenitori con nomi, ora non denotano solo valori numerici (come in C), ma anche veri e propri oggetti irriducibili.

Non ci sono meccanismi per controllo diretto memoria. Le variabili sono nomi "locali" con riferimenti ad'\textit{oggetti} e non maschere di indirizzi in memoria a cui accedere direttamente.

Le variabili posso essere di tipo \textit{Java Types} quindi classi predefinite e autoimplementate oppure \textit{tipi primitivi}.

Visibilità legata al blocco di definizione.

Variabili non inizializzate sono inutilizzabili.

Il \textbf{garbage collector} (componente della JVM) dealloca automaticamente memoria non più utilizzata direttamente o indirettamente dall'heap. Un'oggetto continua a esistere dopo la fine esecuzione dello scope di una variabile che gli fa riferimento.

\subsection{Tipi Primitivi}
Non conviene trattare tutto come oggetto. I tipi atomici del C si sono mantenuti definendo una dimensione fissa e rimuovendo gli unsigned. Si è introdotto boolean con \textbf{true/false}.

Questi tipi sono unici e fissi da linguaggio.

\smallskip
\centerline{
\begin{tabular}{llll}
	\hline
	Tipi  primitivi & Dimensione & Minimo & Massimo \\
	\hline
	boolean & -- & -- & -- \\
	char & 16bits & Unicode 0 & Unicode $2^{16}-1$ \\
	byte & 8bits & -128 & +127 \\
	short & 16bits & $-2^{15}$ & $-2^{15}-1$ \\
	int & 32bits & $-2^{31}$& $+2^{31}-1$ \\
	long & 64bits  & $-2^{63}$& $-2^{63}-1$\\
	float & 32bits & IEEE754 & IEEE754 \\
	double & 64bits & IEEE754 & IEEE754 \\
	\hline
\end{tabular}}
\smallskip

Le librerie \textit{BigDecimal, BigInteger} gestiscono numeri di dimensione/precisione arbitraria.

\textbf{Nota:} In Java l'uso della memoria per i valori true/false di boolean, e tante altre cose, non sono date a sapere al programmatore dato che si dovrebbe concentrare su altro.

\subsection{Stack e Heap}
Gli oggetti sono memorizzati nell'\textbf{heap}. Tutte le variabili sono memorizzate nello \textbf{stack}.

Le variabili di tipo primitivo contengono direttamente il valore. Le variabili tipo classe contengono il riferimento dell'oggetto oppure null.

Nota: Uno stesso oggetto può essere puntato da variabili che si riferiscono alla stessa identità.

\subsection{Oggetti lato Utente}
Dichiarazione, creazione, inizializzazione:

\inline{<Tipo | var> <nome> = <new Tipo([Tipo1 par1, ...]) | altraVariabile | null>;}
\begin{itemize}
	\item \inline{<Tipo> <nome>;} Si può solo dichiarare una variabile oggetto per poi crearla e inizializzarla successivamente.
	\item solo quando si scrive \textit{new} (Keyword di linguaggio) si crea un oggetto dalla classe indicata.
	\item \inline{var}\footnote{Local variable type inference} fa infierire\footnote{Far dedurre al compilatore il tipo della variabile locale dall'espressione assegnata.} il tipo della variabile locale per allegerire il codice. Se manca l'espressione non va, esempio \inline{var i;}.
	\item \inline{altraVariabile} deve essere della stessa classe della variabile che sto definendo.
\end{itemize}







\section{Classi}
Sono template (tipo, struttura in memoria, comportamento) per generare oggetti (istanza).

Le classi hanno un nome (NomeClasse) che sarà anche il nome del tipo per le variabili e del file.

I membri fondamentali di una classe sono:
\begin{itemize}
	\item \textbf{Campi}, descrivono la struttura/stato
	\item \textbf{Metodi}, descrivono i messaggi e il comportamento
\end{itemize}

\multiline{strutturaclasse}{}{CodiciEsempio/StruturaClasse.java}

Definisco le configurazioni.

Le classi sono tipi di dato in un linguaggio a oggetti tutto è un oggetto fino a un certo punto.

\subsection{Campi}
Sono lo stato attuale dell'oggetto.
Simili hai membri di una struttura C, con la differenza che possono essere 0,1,diversi (5-7max). Simili a variabili (tipo+nome), ma non si può usare \textit{var}.
Possono essere valori primitivi o altri oggetti (anche della classe stessa).
L'ordine dei non conta.

I campi sono iniziabili alla dichiarazione dell'oggetto (coi parametri), sennò sono inizializzati in base al tipo a \textbf{0, false, null}.

Uso dei campi lato utente: Assegnamento \inline{... obj.campo = ...}, Lettura \inline{... = obj.campo ...}

\subsection{Metodi}
Definiscono il comportamento dell'oggetto.
Simili a funzioni C. Hanno un intestazione (tipo di ritorno|void, nome, argomenti) e un corpo.
I metodi di una classe possono essere 0, 1, diversi.

i metodi possono leggere/scrivere i campi.

Uso dei metodi lato utente \inline{... obj.metodo() ...}. L'invocazione del metodo, corrisponde a inviare un messaggio al receiver (obj nell'esempio) azionando l'esecuzione del corpo del metodo.

\multiline{strutturametodo}{}{CodiciEsempio/StrutturaMetodo.java}

\subsection{La variabile \textit{this}}
Variabile contenente il riferimento all'oggetto che sta gestendo il messaggio corrente. Si usa per rendere meno ambiguo il codice accedento tramite \textit{this} a campi o metodi. (Usare sempre).
\inline{... this.cmapo ... this.metodo() ...}

\subsection{Precisazioni}
Inizializzazioni particolari degli oggetti Stringa:
\begin{enumerate}
	\item \inline{... = new String();}, stringa vuota (è diverso da null).
	\item \inline{... = "..."}, come in C, comportamento speciale degli oggetti Stringa.
\end{enumerate}

\end{document}