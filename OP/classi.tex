\section{Classi}
Sono template (tipo, struttura in memoria, comportamento) per generare oggetti (istanza).

Le classi hanno un nome (NomeClasse) che sarà anche il nome del tipo per le variabili e del file.

I membri fondamentali di una classe sono:
\begin{itemize}
	\item \textbf{Campi}, descrivono la struttura/stato
	\item \textbf{Metodi}, descrivono i messaggi e il comportamento
\end{itemize}
Poi possiamo avere Costruttori, ecc.

Le classi sono tipi di dato in un linguaggio a oggetti tutto è un oggetto fino a un certo punto.

\multiline{strutturaclasse}{}{CodiciEsempio/StruturaClasse.java}

\subsection{Campi}
Sono lo stato attuale dell'oggetto. Nome con la minuscola.
Simili hai membri di una struttura C, con la differenza che possono essere 0,1,diversi (5-7max). Simili a variabili (tipo+nome), ma non si può usare \textit{var}.
Possono essere valori primitivi o altri oggetti (anche della classe stessa).
L'ordine dei non conta.

I campi sono iniziabili alla dichiarazione dell'oggetto (coi parametri), sennò sono inizializzati in base al tipo a \textbf{0, false, null}. Mai alla dichiarazione si inizalizzano (...=...;).

\oneline{public|private|"" [final] <tipo> <nomeCampo>;}

Uso dei campi lato utente: Assegnamento \inline{... obj.campo = ...}, Lettura \inline{... = obj.campo ...}

\subsection{final e costanti}
Modificatore (da usare quando si può) per campi, variabili e parametri. Dopo inizializzazione non più modificabile il valore.

Per variabili di tipo classe non si può cambiare il riferimento all'oggetto, ma i campi dell'oggetto si.

\oneline{<tipo> final <nomeVar [= new ... | valore;]}

Per fare le \textbf{costanti} (NOME\_COSTANTE) usare \inline{static final ...;}. Soluzione a \textit{Magic Number}.

\subsubsection{Oggetti immutabili}
Oggetti i cui campi non si modificheranno mai, dopo la prima modifica del costruttore.

Si creano dichiarando \inline{final private} tutti i campi della classe di tipo: tipi primitivi o oggetti immutabili.

\subsection{Metodi}
Definiscono il comportamento dell'oggetto. Nome con la minuscola.
Simili a funzioni C. Hanno un intestazione (tipo di ritorno|void, nome, argomenti) e un corpo.
I metodi di una classe possono essere 0, 1, diversi.

i metodi possono leggere/scrivere i campi.

Uso dei metodi lato utente \inline{... obj.metodo() ...}. L'invocazione del metodo, corrisponde a inviare un messaggio al receiver (obj nell'esempio) azionando l'esecuzione del corpo del metodo.

\multiline{strutturametodo}{}{CodiciEsempio/StrutturaMetodo.java}

\subsubsection{return this}
Un metodo può ritorna \inline{this}, l'oggetto corrente. Deve avere come tipo di ritorno la classe stessa.

Usato per concatenare più metodi in una sola espressione, combinando i risultati parziali, e restituendo con l'ultimo metodo l'oggetto processato:

\inline{... obj.m1().m2()...mThis();}

Questo schema è detto \textbf{fluente}.

\subsection{Costruttore}
Simile a un metodo con nome il nome della classe, senza tipo di ritorno e opzionalmente dei parametri formali.

\inline{NomeClasse([parametri]) {...}}

Se non si definisce il costruttore viene implicitamente inserito il costruttore di default (0 parametri) che inizializza i campi ai tipi di default.

\subsubsection{Overloading}
Si possono definire più costruttori e metodi, ma non è una buona pratica.
Nel caso devono essere \textbf{distinguibili dal numero e/o tipo} dei parametri.

I costruttori si possono \textbf{invocare a vicenda} (riuso codice). \inline{this(...)} deve essere la prima riga del corpo del costruttore.

\multiline{riusocostruttori}{}{CodiciEsempio/RiusoCostruttori.java}

\subsection{this}
Variabile contenente il riferimento all'oggetto che sta gestendo il messaggio corrente. Si usa per rendere meno ambiguo il codice accedento tramite \textit{this} a campi o metodi. (Usare sempre).
\inline{... this.cmapo ... this.metodo() ...}

\subsection{static}
Scollega metodi e campi dagli oggetti di una stessa classe, rendendoli condivisi tra essi. I metodi diventano funzioni pure e i campi variabili globali alla classe.

Si possono fare classi solo statiche (utility class, NomeClasse+s) più non statiche (approccio migliore) o miste (campi e metodi static in fondo).

Richiamo fuori dalla classe: \inline{... nomeClase.campo|metodo() ...}.
Richiamo nella classe come normale campo o metodo.
Richiamabile anche da un oggetto.

\oneline{static campo|metodo|classe}

Quando \inline{static}?
Si fanno statici i metodi o campi che sono indipendenti dallo stato di un oggetto della classe. Fanno un servizio indipendente dal singolo oggetto.

\subsection{Livelli d'accesso}
Si antepongono a classi, metodi, campi, costruttori per definire il grado di utilizzo, in ordine crescente di libertà:
\begin{enumerate}
	\item \inline{private} Visibile e richiamabile solo nella classe di definizione.
	\item \inline{package} default (no keyword). Visibile e richiamabili dentro il package e invisibili fuori.
	\item \inline{protected} Visibile alla classe corrente e dalle sotto classi ricorsivamente. (in java anche al package).
	\item \inline{public} Visibile e richiamabile da qualunque classe.
\end{enumerate}

In una unità di compilazione una sola classe è \inline{public} e con lo stesso nome del file.

Vantaggio: Si può far rispettare al meglio il contratto\footnote{Insieme degli scienari d'utilizzo e quindi aspettative di un utente al suo utilizzo.} di un oggetto.

\subsubsection{Incapsulamento}
\begin{center}
	Scopo: Avere controllo su come vengono usati i dati.
\end{center}

\colTwo{Dichiaro \inline{public} solo metodi/costruttori, di "design", necessari all'utente per interagire/creare l'istanza della classe.}{Dichiaro \inline{private} tutti i metodi/costruttori di "implementazione", e \textbf{tutti i campi}.}{45}{45}

Cosi modifiche implementative non influenzano uso dell'utente. Controllo dati e riuduco possibilità errori lato utente. Maggior distacco tra dato e implementazione. (\textit{Information hidding} = nascondere implementazione informazioni).

\bigskip

\textbf{Principio di decomposizione (\textit{divide et impera})}: Soluzione di un problema complesso è la somma di due o più sotto problemi più semplici, tra loro indipendenti.

Tali sotto problemi devono avere il minor numero di dipendenze\footnote{Riferito alle classi è quando una classe usa al suo interno un'altra. Ciò può comportare modifiche a cascata rischiando la \textit{sindrome dell'"intoccambilità"}.} reciproche. Permettendo maggior autonomia, più modifiche senza danni collaterali e meno interazioni.

Una buona divisione da moduli con basse dipendenze esterne e alte dipendenze interne.

Nella OO le suddivisioni base sono pacakge, classi e metodi.

\subsection{toString()}
Ogni classe deve definire tale metodo che restituisce una rappresentazione in stringa dell'oggetto, utile in fase di debug.

Automaticamente chiamato dall'operatore "+" nelle concatenazioni tra stringhe e oggetti.

\subsection{Precisazioni}
Inizializzazioni particolari degli oggetti Stringa:
\begin{enumerate}
	\item \inline{... = new String();}, stringa vuota (è diverso da null).
	\item \inline{... = "..."}, come in C, comportamento speciale degli oggetti Stringa.
\end{enumerate}

\subsection{Fasi Implementazione}
\begin{enumerate}
	\item Progettazione della parte pubblica della classe (nome classe, signatura metodi e costruttori necesari).
	\item Costruzione dello stato (campi privati, con nome diverso dai metodi relativi).
	\item Completamento implementazione. (test).
	\item Miglioramento codice finale (commenti, eliminare \textit{Magic Number}, fattorizzare sotto-funzioni). (test).
	\item Test del risultato con test-case. (test-driven development con JUnit).
\end{enumerate}

\section{Oggetti lato Utente}
Dichiarazione, creazione, inizializzazione:

\inline{<Tipo | var> <nome> = <new Tipo([argomenti]) | altraVariabile | null>;}
\begin{itemize}
	\item \inline{<Tipo> <nome>;} Si può solo dichiarare una variabile oggetto per poi crearla e inizializzarla successivamente.
	\item solo quando si scrive \inline{new} (Keyword di linguaggio di chiamata al costruttore) si crea e inizializza un oggetto dalla classe indicata. \inline{new} da il riferimento (\inline{this}) dell'oggetto alla variabile.
	\item \inline{var}\footnote{Local variable type inference.} fa infierire\footnote{Far dedurre al compilatore il tipo della variabile locale dall'espressione assegnata.} il tipo della variabile locale per allegerire il codice. Se manca l'espressione non va, esempio \inline{var i;}.
	\item \inline{altraVariabile} deve essere della stessa classe della variabile che sto definendo.
\end{itemize}