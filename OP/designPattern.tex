\section{Design Pattern}
Approccio fondamentale:

\begin{center}
	\begin{tikzcd}
		Interfacce               &  & {} \arrow[ll, "concetto"', no head, maps to] \\
		ClassiAstratte \arrow[u] &  & {} \arrow[ll, "condiceComune"', no head, maps to]              \\
		ClassiConcrete \arrow[u] &  & {} \arrow[ll, "implementazioni"', no head, maps to]
	\end{tikzcd}
\end{center}

Variabili, argomenti, tipi di ritorno sono interfacce mentre le \inline{new} sono solo di classi Concerete.

\begin{itemize}
	\item \textbf{Iteratore (Iterator)}: Oggetto usato per accedere ad'una sequenza d'elementi (presi da una sorgente di dati).
	\multiline{}{}{CodiciEsempio/iterator.java}


\end{itemize}
