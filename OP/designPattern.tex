\section{Design Pattern}
Approccio fondamentale:

\begin{center}
	\begin{tikzcd}
		Interfacce               &  & {} \arrow[ll, "concetto"', no head, maps to] \\
		ClassiAstratte \arrow[u] &  & {} \arrow[ll, "condiceComune"', no head, maps to]              \\
		ClassiConcrete \arrow[u] &  & {} \arrow[ll, "implementazioni"', no head, maps to]
	\end{tikzcd}
\end{center}

Variabili, argomenti, tipi di ritorno sono interfacce mentre le \inline{new} sono solo di classi Concerete.

\begin{enumerate}
	\item \textbf{Iterator}: Oggetto usato per accedere ad'una sequenza d'elementi (presi da una sorgente di dati).
	\multiline{}{}{CodiciEsempio/iterator.java}
	\item ne manca uno...
	\item \textbf{Strategy}: Un oggetto A delega a un oggetto B l'implementazione d'una strategia che poi sarà usata dall'oggetto A tramite l'oggetto B.
	\item \textbf{Observer}: Oggetto osservatore si registra (attach) alla sorgente degli eventi per reagire a un evento specifico. Quando tale evento avviene la sorgente aziona (notify) l'osservatore che reagisce con un comportamente programmato.

	\begin{figure}[h]
		\centering
		\includegraphics[width=0.5\textwidth]{ObserverPattern.png}
	\end{figure}
\end{enumerate}
