\section{Strumenti}

\subsection{JUnit}
Crea classi di test tramite metodi annotati che testano scenari per asserire correttenza risultato rispetto a valore atteso.

\multiline{}{}{CodiciEsempio/test.java}

Metodi di asserzione: \inline{asserTrue(), assertEquals(), assertFalse(), assertThorws()}

Un test va a buon fine se non fallisce nessun assert (o è interrotto da \inline{return}).
Per far fallire il test secondo una "nostra logica":\inline{fail()}.

Per far eseguire un test per primo si usa l'annotazione \inline{@BeforeEach}.

\subsubsection{Esecuzione}
Eseguire il build comporta il check (controlli+test(compileTestJava+compileJava)) e asseble (jar+compileJava).

\oneline{./gradlew build javadoc}

Le cartelle principali sono: \inline{/build, /src/main, /src/test}.

Configurazione Gradle (\inline{build.gradle.kts}) per Junit slude 08 pag7-8.

\subsection{JAR}
JAR (Java ARchive) è il confezionamento di un progetto java (classi, risorse, Manifest) in un'unico file (ZIP).

\textbf{Manifest}, \inline{META-INF/MANIFEST.MF}, è un file descrittivo dell'applicazione (classe con main del programma).

Il JAR può essere eseguito da \inline{java} (se associato dall'os) e usato come libreria per altri progetti.

\bigskip

Un file JAR si crea con \inline{jar} e le opzioni \inline{c} (intenzione creare file jar), \inline{f} (indica file output, altrimenti default strout), \inline{m} (specifca MANIFEST, altrimenti default senza main start).

\oneline{jar [c][f][m] nomeJar.jar [fileMANIFEST] [filesJava]}

Per eseguire un JAR da CLI.

\oneline{java -jar nomeJar.jar}

Configurazione Gradle (\inline{build.gradle.kts}) per Fat-JAR slude 08 pag16-.

\subsubsection{Documentazione semi-automatica}
Usando \inline{javadoc} e una sintassi dei commenti si crea automaticamente della documentazione HTML java.

Sintassi:
I commenti del tipo \inline{/** ... */} e in testa a classi/interfacce/membri, sono processati

Tag informativi:
\begin{itemize}
	\item \inline{@param} Per costruttori/metodi/classi parametrice, ne descrive un parametro.
	\item \inline{@type-parameters} Per classi e metodi generici, ne descrive un parametro.
	\item \inline{return} Per metodi, ne descrive il valore di ritorno.
	\item  \inline{@throws} Per costruttori/metodi, ne descrive le eccezzioni che possono essere lanciate e il loro motivo.
	\item \inline{@see} Linka altre entità che potrebbero interessare.
	\item  \inline{@deprecated} Da usare per entità deprecati (\inline{@Deprecated}).
	\item \inline{{@link [nameClass]\#method}} Link ipertestuali a entità della stesa classe o esterne.
	\item  \inline{{@code ...}} Formatta il testo monospazionandolo.
	\item \inline{{inheritDoc}} Copia documentazione entità sovrascritta. Da usare se uso esterno rimane invariato.
\end{itemize}