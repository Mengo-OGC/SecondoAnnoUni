\section{Struttura Programma Java}
Un programma Java è composta da librerie di classi del JDK, Package\footnote{Contenitori, gerarcici tra loro, di una decina di classi di alto livello con scopo comune} e Moduli\footnote{Insieme di Package costituente un frammento di codice autonomo.}, librerie di esterne e un insieme di classi fondamentali.

La classe cardine di un programma è la class main\footnote{Un main è il punto d'accesso di un programma.}.
\multiline{main}{}{CodiciEsempio/main.java}
Il parametro del main è un array di stringhe che sono i parametri che l'utente può immettere da tastiera quando il programma è lanciato da CLI. Poco usato.

\subsection{Esecuzione Programma}
\begin{enumerate}
	\item Salvare la classe in un file \textbf{"NomeClasse.java"}.
	\item Compilare con \textit{javac NomeFileClasse.java}. Genererà il \textbf{bytecode NomeFileClasse.class} per la JVM.
	\item Esegiure con \textit{java NomeFileClasse}. La JVM cercherà il main da cui partire a eseguire.
\end{enumerate}
Lavorando con più file: si compila tutto con \textit{javac *.java} poi si esegue solo la classe main.

\subsection{I package}
Per prassi nella cartella di un progetto, si fa corrispondere hai package auto-costruiti delle directory del file system.

Di default il package di una unità di compilazione\footnote{File.java compilabile atomicamente contenente classi una dopo l'altra.}, ".java", è la root di gerarchia.
Per definire il pakage di appartenenza si specifica a inizio sorgente:

\inline{package pname;}

Per poi usare una classe di un package bisogna importarla (non aggiunge codice nell'eseguibile):
\begin{itemize}
	\item \inline{import java...;} importa una singola classe.
	\item \inline{import java...*;} importa l'intero Package.
	\item \inline{import java.lang.*;} importazione di default.
\end{itemize}
Il nome completo di una classe dipende dal Package in cui si trova. Se non si importa bisogna specificare all'uso il nome completo.

\subsection{Convenzioni}
\begin{itemize}
	\item Linee max 90 caratteri e indenzazione 2-4 caratteri.
	\item Una sola istruzione per riga e quindi si definisce una sola variabile per riga, quando necessario, sempre inizializzata.
	\item Apertura "\{" a fine riga della dichiarazione e chiusura "\}" in linea a dove inizia la riga di apertura.
	\item Condice senza righe vuote di separazione a meno di metodi/costruttori.
	\item Non usare assegnamenti dentro espressioni e usare parentesi solo in espressioni non banali.
	\item Nomi classi e interfacce in "PascalCase". Nomi metodi, campi, variabili in "camelCase" e costanti in "BIG\_CASE".
	\bigskip
	\item I metodi che returnano, senza parametri, si chiamano \inline{get...} oppure \inline{is/has...} se return boolean.
	\item I metodi che returnano void, e accettano parametri, si chiamano \inline{set...}.
\end{itemize}

\subsubsection{I Commenti}
Oltre a \inline{//... e /*...*/} in java abbiamo \inline{/** ... */} un commento multi linea usato per generare documentazione.


\section{(Almost) Everything is an object}
Le variabili, contenitori con nomi, ora non denotano solo valori numerici (come in C), ma anche veri e propri oggetti irriducibili.

Non ci sono meccanismi per controllo diretto memoria. Le variabili sono nomi "locali" con riferimenti ad'\textit{oggetti} e non maschere di indirizzi in memoria a cui accedere direttamente.

Le variabili posso essere di tipo \textit{Java Types} quindi classi predefinite e autoimplementate oppure \textit{tipi primitivi}.

Visibilità legata al blocco di definizione.
Variabili non inizializzate sono inutilizzabili.

\subsection{Stack e Heap}
Gli oggetti sono memorizzati nell'\textbf{heap}. Tutte le variabili sono memorizzate nello \textbf{stack}.

Le variabili di tipo primitivo contengono direttamente il valore. Le variabili tipo classe contengono il riferimento dell'oggetto oppure null.

Nota: Uno stesso oggetto può essere puntato da variabili che si riferiscono alla stessa identità.

\subsection{Tipi Primitivi}
Ripasso: Un tipo classifica valori/oggetti tramite un nome, set valori, operatori.

I tipi atomici \textbf{signed} del C si sono mantenuti (non conveniva trattarli come oggetti) definendo un unica interpretazione:
\begin{center}
	\begin{tabular}{llll}
		\hline
		Tipi  primitivi & Dimensione & Minimo & Massimo \\
		\hline
		boolean & -- & -- & -- \\
		char & 16bits & Unicode 0 & Unicode $2^{16}-1$ \\
		byte & 8bits & -128 & +127 \\
		short & 16bits & $-2^{15}$ & $-2^{15}-1$ \\
		int & 32bits & $-2^{31}$& $+2^{31}-1$ \\
		long & 64bits  & $-2^{63}$& $-2^{63}-1$\\
		float & 32bits & IEEE754 & IEEE754 \\
		double & 64bits & IEEE754 & IEEE754 \\
		\hline
	\end{tabular}

	{\tiny Le librerie \textit{BigDecimal, BigInteger} gestiscono numeri di dimensione/precisione arbitraria.}
\end{center}

\textbf{typing statico}: espressioni tipo noto al compilatore, vantaggio intercettazione errori.

\textbf{Nota:} Uso memoria non dato a sapere al programmatore.

\subsubsection{Boolean}
Introdotto come tipo risultato delle espressioni e condizioni.
\begin{itemize}
	\item \textbf{Valori} true, false.
	\item \textbf{Operatori Unari} \inline{! (not)}. Peratori binari: \inline{\& (and), | (or), \^ (xor), \&\& (and-C), || (or-c)}\footnote{\& e | valutano sempre primo e secondo termine, essendo pensati per operazioni bit a bit mentre \&\& e || valotano il secondo operatore solo se necessario.}.
	\item \textbf{Operatori confronto numerici} \inline{<, >, <=, >=}.
	\item \textbf{Operatori di uguaglianza} \inline{==, !=}. Con gli oggett di base confronta i riferimenti.
	\item \textbf{Operatore ternario} \inline{<eb> ? e1 : e2}. restituisce e1 se eb è true, altrimenti restituisce e2. e1 e e2 espressioni dello stesso tipo.
\end{itemize}

\subsubsection{Caratteri}
Rappresentazione \inline{'carattere'} o \inline{'\u<0-65535>'}. Formato ASCII o UTF16 (16bit). I caratteri d'escape si fanno \inline{'\\<specificatore>'}.

\subsubsection{Interi}
\begin{itemize}
	\item \textbf{Codificati} in complemento a 2.
	\item \textbf{Operatori} \inline{+ ,-, *, /, \%, + e - unari, \&, |, ~, <<, >>, >>>}. Output stesso tipo Input.
	\item \textbf{Rappresentabili} in codifica decimale (1\_000\_000), ottale (0...), esadecimale (0x...).
	\item \textbf{Input tastiera} default int (più usato e efficente). Se voglio un long va aggiunta una "l" dopo il numero.
\end{itemize}

\subsubsection{Virgola mobile}
\begin{itemize}
	\item \textbf{Operatori} \inline{+ ,-, *, /, \%, + e -} unari.
	\item \textbf{Codificati} in IEEE754.
	\item \textbf{Rappresentabili} in codifica decimale o scientifica.
	\item \textbf{Input tastiera} default double. Se voglio un float (più efficente) va aggiunta una "f" dopo il numero.
\end{itemize}
\textbf{Ricorda}: IEEE754 ha errori di precisione che portano all'approssimazione dei risultati.

\subsection{Conversioni}
\begin{itemize}
	\item \textbf{Implicita} automaticamente applicata nelle esperessioni e nell'assegnamento portando tutti tipi a quello più "generale" presente nell'espressione. Detta \textit{coercizione}.

	$byte\to short \to int \to long \to float \to double$
	\item \textbf{Esplicito} fatto con operatore di casting: \inline{...=(<tipo>)<espressione>;}. Può causare perdità di informazioni.
\end{itemize}

\subsection{Array}
Oggetti (variabili hanno riferimento nello heap) di lunghezza esplicita e acessibile, inacessibile fuori dai limiti (errore esecuzione), mutabili. Indirizzi elementi [0, lunghezza-1].

Sintassi:
\begin{itemize}
	\item \inline{<tipo>[] <nome> = new <tipo>[] \{v1,...,vn\};}
	\item \inline{<tipo>[] <nome> = new <tipo>[<dim>];}. Elementi inizializzati a valore default tipo (false, 0, null).
	\item \inline{<tipo>[] <nome> = \{v1,...,vn\};}
\end{itemize}

Note: Si possono fare array di array. Gli array di oggetti sono oggetti con puntatori ad'altri oggetti.

Accesso a variabile \inline{... <nome>[<ind>] ...}.

Per sapere la lunghezza dell'array \inline{<nome>.length}.

\subsubsection{Foreach}
Astazione \inline{for} in sola \textbf{lettura}, utile quando non importa ordine scorrimento elementi collezione e valore indice.

\oneline{for(final<tipo> <v>: <e>)}

\begin{itemize}
	\item \inline{v} è variabile del tipo indicato che vale via via ogni elemento della colezzione.
	\item \inline{e} è un espressione che restituisce una collezione di elementi del tipo indicato.
	\item Si può esplicitare \inline{final}, anche se non servirebbe, perchè il foreach crea una nuova variabile a ogni iterazione in realtà.
\end{itemize}






\section{Classi}
Sono template (tipo, struttura in memoria, comportamento) per generare oggetti (istanza).

Le classi hanno un nome (NomeClasse) che sarà anche il nome del tipo per le variabili e del file.

I membri fondamentali di una classe sono:
\begin{itemize}
	\item \textbf{Campi}, descrivono la struttura/stato
	\item \textbf{Metodi}, descrivono i messaggi e il comportamento
\end{itemize}
Poi possiamo avere Costruttori, ecc.

Le classi sono tipi di dato in un linguaggio a oggetti tutto è un oggetto fino a un certo punto.

\multiline{strutturaclasse}{}{CodiciEsempio/StruturaClasse.java}

\subsection{Campi}
Sono lo stato attuale dell'oggetto. Nome con la minuscola.
Simili hai membri di una struttura C, con la differenza che possono essere 0,1,diversi (5-7max). Simili a variabili (tipo+nome), ma non si può usare \textit{var}.
Possono essere valori primitivi o altri oggetti (anche della classe stessa).
L'ordine dei non conta.

I campi sono iniziabili alla dichiarazione dell'oggetto (coi parametri), sennò sono inizializzati in base al tipo a \textbf{0, false, null}. Mai alla dichiarazione si inizalizzano (...=...;).

\oneline{public|private|"" [final] <tipo> <nomeCampo>;}

Uso dei campi lato utente: Assegnamento \inline{... obj.campo = ...}, Lettura \inline{... = obj.campo ...}

\subsection{final e costanti}
Modificatore (da usare quando si può) per campi, variabili e parametri. Dopo inizializzazione non più modificabile il valore.

\oneline{<tipo> final <nomeVar [= new ... | valore;]}

Per fare le \textbf{costanti} (NOME\_COSTANTE) usare \inline{static final ...;}. Soluzione a \textit{Magic Number}.

\subsubsection{Oggetti immutabili}
Oggetti i cui campi non si modificheranno mai, dopo la prima modifica del costruttore.

Si creano dichiarando \inline{final private} tutti i campi della classe di tipo: tipi primitivi o oggetti immutabili.

\subsection{Metodi}
Definiscono il comportamento dell'oggetto. Nome con la minuscola.
Simili a funzioni C. Hanno un intestazione (tipo di ritorno|void, nome, argomenti) e un corpo.
I metodi di una classe possono essere 0, 1, diversi.

i metodi possono leggere/scrivere i campi.

Uso dei metodi lato utente \inline{... obj.metodo() ...}. L'invocazione del metodo, corrisponde a inviare un messaggio al receiver (obj nell'esempio) azionando l'esecuzione del corpo del metodo.

\multiline{strutturametodo}{}{CodiciEsempio/StrutturaMetodo.java}

\subsubsection{return this}
Un metodo può ritorna \inline{this}, l'oggetto corrente. Deve avere come tipo di ritorno la classe stessa.

Usato per concatenare più metodi in una sola espressione, combinando i risultati parziali, e restituendo con l'ultimo metodo l'oggetto processato:

\inline{... obj.m1().m2()...mThis();}

Questo schema è detto \textbf{fluente}.

\subsection{Costruttore}
Simile a un metodo con nome il nome della classe, senza tipo di ritorno e opzionalmente dei parametri formali.

\inline{NomeClasse([parametri]) {...}}

Se non si definisce il costruttore viene implicitamente inserito il costruttore di default (0 parametri) che inizializza i campi ai tipi di default.

\subsubsection{Overloading}
Si possono definire più costruttori e metodi, ma non è una buona pratica.
Nel caso devono essere \textbf{distinguibili dal numero e/o tipo} dei parametri.

I costruttori si possono \textbf{invocare a vicenda} (riuso codice). \inline{this(...)} deve essere la prima riga del corpo del costruttore.

\multiline{riusocostruttori}{}{CodiciEsempio/RiusoCostruttori.java}

\subsection{this}
Variabile contenente il riferimento all'oggetto che sta gestendo il messaggio corrente. Si usa per rendere meno ambiguo il codice accedento tramite \textit{this} a campi o metodi. (Usare sempre).
\inline{... this.cmapo ... this.metodo() ...}

\subsection{static}
Scollega metodi e campi dagli oggetti di una stessa classe, rendendoli condivisi tra essi. I metodi diventano funzioni pure e i campi variabili globali alla classe.

Si possono fare classi solo statiche (utility class, NomeClasse+s) più non statiche (approccio migliore) o miste (campi e metodi static in fondo).

Richiamo fuori dalla classe: \inline{... nomeClase.campo|metodo() ...}.
Richiamo nella classe come normale campo o metodo.
Richiamabile anche da un oggetto.

\oneline{static campo|metodo|classe}

Quando \inline{static}?
Si fanno statici i metodi o campi che sono indipendenti dallo stato di un oggetto della classe. Fanno un servizio indipendente dal singolo oggetto.

\subsection{Livelli d'accesso}
Si antepongono a classi, metodi, campi, costruttori, ecc per definire il grado di utilizzo:
\begin{itemize}
	\item \inline{public} Visibile e richiamabile da qualunque classe.
	\item \inline{package} default (no keyword). Visibile e richiamabili dentro il package e invisibili fuori.
	\item \inline{private} Visibile e richiamabile solo nella classe di definizione.
\end{itemize}

In una unità di compilazione una sola classe è \inline{public} e con lo stesso nome del file.

Vantaggio: Si può far rispettare al meglio il contratto\footnote{Insieme degli scienari d'utilizzo e quindi aspettative di un utente al suo utilizzo.} di un oggetto.

\subsubsection{Incapsulamento}
\colTwo{Dichiaro \inline{public} solo metodi/costruttori, di "design", necessari all'utente per interagire/creare l'istanza della classe.}{Dichiaro \inline{private} tutti i metodi/costruttori di "implementazione", e \textbf{tutti i campi}.}{45}{45}

Cosi modifiche implementative non influenzano uso dell'utente. Controllo dati e riuduco possibilità errori lato utente. Maggior distacco tra dato e implementazione. (\textit{Information hidding}).

\bigskip

\textbf{Principio di decomposizione (\textit{divide et impera})}: Soluzione di un problema complesso è la somma di due o più sotto problemi più semplici, tra loro indipendenti.

Tali sotto problemi devono avere il minor numero di dipendenze\footnote{Riferito alle classi è quando una classe usa al suo interno un'altra. Ciò può comportare modifiche a cascata rischiando la \textit{sindrome dell'"intoccambilità"}.} reciproche. Permettendo maggior autonomia, più modifiche senza danni collaterali e meno interazioni.

Una buona divisione da moduli con basse dipendenze esterne e alte dipendenze interne.

Nella OO le suddivisioni base sono pacakge, classi e metodi.

\subsection{toString()}
Ogni classe deve definire tale metodo che restituisce una rappresentazione in stringa dell'oggetto, utile in fase di debug.

Automaticamente chiamato dall'operatore "+" nelle concatenazioni tra stringhe e oggetti.


\subsection{Precisazioni}
Inizializzazioni particolari degli oggetti Stringa:
\begin{enumerate}
	\item \inline{... = new String();}, stringa vuota (è diverso da null).
	\item \inline{... = "..."}, come in C, comportamento speciale degli oggetti Stringa.
\end{enumerate}

\subsection{Fasi Implementazione}
\begin{enumerate}
	\item Progettazione della parte pubblica della classe (nome classe, signatura metodi e costruttori necesari).
	\item Costruzione dello stato (campi privati, con nome diverso dai metodi relativi).
	\item Completamento implementazione. (test).
	\item Miglioramento codice finale (commenti, eliminare \textit{Magic Number}, fattorizzare sotto-funzioni). (test).
	\item Test del risultato con test-case. (test-driven development con JUnit).
\end{enumerate}

\section{Oggetti lato Utente}
Dichiarazione, creazione, inizializzazione:

\inline{<Tipo | var> <nome> = <new Tipo([argomenti]) | altraVariabile | null>;}
\begin{itemize}
	\item \inline{<Tipo> <nome>;} Si può solo dichiarare una variabile oggetto per poi crearla e inizializzarla successivamente.
	\item solo quando si scrive \inline{new} (Keyword di linguaggio di chiamata al costruttore) si crea e inizializza un oggetto dalla classe indicata. \inline{new} da il riferimento (\inline{this}) dell'oggetto alla variabile.
	\item \inline{var}\footnote{Local variable type inference} fa infierire\footnote{Far dedurre al compilatore il tipo della variabile locale dall'espressione assegnata.} il tipo della variabile locale per allegerire il codice. Se manca l'espressione non va, esempio \inline{var i;}.
	\item \inline{altraVariabile} deve essere della stessa classe della variabile che sto definendo.
\end{itemize}