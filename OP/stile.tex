\usepackage[italian]{babel}
\usepackage{listings}
\usepackage{xcolor}
\usepackage{fancyhdr}
\usepackage[colorlinks=true, linkcolor=black]{hyperref}

\renewcommand{\lstlistingname}{Codice}		% titoletto dei caption

\lstdefinestyle{mystyle}{
	basicstyle=\ttfamily\small,				% stile testo leggermente più piccolo e monospace
	commentstyle=\itshape\color{gray},		% commenti grigi in italico
	keywordstyle=\color{blue},				% parole chiave linguaggio in blu
	stringstyle=\color{green!40!black},		% stringhe in "verde"
	numbers=left,							% numeri riga visibili sulla sinistra
	numberstyle=\tiny\color{gray},			% numeri riga piccolissimi e grigi
	breaklines=true,						% spezza righe troppo lunghe su più righe
	showspaces=false,						% seguenti 3 comandi nascondono caratteri indesiderati
	showstringspaces=false,
	showtabs=false,
	tabsize=2,								% larghezza dei tab 2 spazi
	xleftmargin=\parindent					% allinea blocco ai paragrafi
}

\lstset{style=mystyle}						% applico stile definito

\newcommand{\multiline}[3]{\lstinputlisting[language=Java, caption={#2}, label={#1}]{#3}}
\newcommand{\inline}[1]{\noindent\lstinline[language=Java]@#1@}
\newcommand{\oneline}[1]{\medskip\inline{#1}\medskip}

\newcommand{\colTwo}[4]{
	\begin{minipage}[t]{0.#3\textwidth}
		#1
	\end{minipage}
	\hfill
	\begin{minipage}[t]{0.#4\textwidth}
		#2
	\end{minipage}\\[0.01\textheight]
}

\pagestyle{fancy}
\fancyhf{}
\fancyhead[L]{\hyperref[toc]{$\leftarrow$Indice}}
\fancyhead[R]{\thepage}