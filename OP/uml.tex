\section{UML (Class Diagram)}
Linguaggio grafico, OO-based, di modellazione di software.
\begin{itemize}
	\item \textbf{Rettangoli}: un box rettangolare per classe o interfaccia diviso in:
	\begin{enumerate}
		\item Nome classe. Nel caso delle interfacce aggiunge \inline{<<interface>>} a sinistra del nome.
		\item Campi (solo per classi). Signatura: \inline{nome : tipo}.
		\item Metodi e costruttori. Signatura: \inline{nome(arg1:tipo,...):tipoRitorno}.
	\end{enumerate}
	- indica \inline{private}, + indica \inline{public}, \# indica \inline{protected} e sottolineato indica \inline{static}.

	Metodi e classi astratte si scrivono in corsivo.

	Metodi e classi \inline{final} (non overraidabili) si indica l'attributo \{leaf\}.

	Per indicare l'uso dei genenrics si indica in un rettangolo tratteggiato il/i generics, dentro il rettangolo, del nome e poi si usano come tipi.
	\item Dipendenze tra classi:
	\begin{itemize}
		\item Composizione è arco con rombo.
		\item Associazione è arco con freccia.
		\item Generalizzazione è arco tratteggiato con triangolo vuoto, raggruppabile.
		\item Estensione è arco con triangolo pieno, raggruppabile.

		Nella classe figlia si specificano solo le aggiunte e ovveride.
	\end{itemize}
	L'arco può essere indicato con le molteplicità: 1,0, n, 0...1, 1...n.

	\item Spesso accompagnato da descrizione testuale.
\end{itemize}

Spesso si omette tutto ciò che non è "design" e le signature ed'alcune relazioni.
