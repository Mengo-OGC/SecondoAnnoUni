\section{UML (Class Diagram)}
Linguaggio grafico, OO-based, di modellazione di software.
\begin{itemize}
	\item \textbf{Classi}: un box rettangolare per classe diviso in:
	\begin{enumerate}
		\item Nome classe. Nel caso delle interfacce aggiunge "<<interface>>" a sinistra del nome.
		\item Campi.
		\item Metodi e costruttori. Signatura: \inline{nome(arg1:tipo,...):tipoRitorno}.
	\end{enumerate}
	- indica \inline{private}, + indica \inline{public} e sottolineato indica \inline{static}.
	\item Dipendenze tra classi:
	\begin{itemize}
		\item Composizione è arco con rombo.
		\item Associazione è arco con freccia.
		\item Generalizzazione è arco tratteggiato con triangolo.
	\end{itemize}
	L'arco può essere indicato con le molteplicità: 1,0, n, 0...1, 1...n.
\end{itemize}

Spesso si omette tutto ciò che non è "design" e le signature ed'alcune relazioni.
