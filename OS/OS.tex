\documentclass{article}
\usepackage[a4paper, hmargin=15mm]{geometry}
\usepackage{tabularx}

\title{Sistemi Operativi}
\author{Leonardo Mengozzi}
\date{}

\newcommand{\sezione}[1]{\hline#1 & Descrizione\\\hline}

\begin{document}
	\maketitle

	\centering
	{\large Comandi bath}\\[0.01\textheight]
	\begin{tabularx}{\textwidth}{lX}
	\sezione{Directory}
	cd \textit{percorso} & Sposta logicamente in una diversa directory, secondo un path asssoluto o relativo.\\
	ls & Visualizza i files/direcotry contenuti nella directory corrente (anche ., ..). Parametri utili:

	\textit{-l} mostra informazioni aggiuntive sui files.\\
	pwd & Visualizza il percorso assoluto, da / fino alla directory corrente.\\

	\sezione{Scripting}
	echo \textit{testo} & Visualizza a video la sequenza di caratteri passata fino al primo "INVIO".\\
	\ & Disabila interpretazione per il carattere successivo, andata a capo, permettendo di stamparlo.\\
	echo "\textit{testo}" & Come echo, ma disabilita interpretazione caratteri speciali e andate a capo.\\
	; & Separatore di più comandi lanciati durante la stessa digitazione.\\

	\sezione{Vari}
	strace \textit{comando} & Dice la sistem-call usata.\\
	\hline
	\end{tabularx}

\end{document}