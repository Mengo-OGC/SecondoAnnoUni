\documentclass{article}
\usepackage[a4paper, hmargin=15mm]{geometry}
\usepackage{tabularx}

\title{Sistemi Operativi}
\author{Leonardo Mengozzi}
\date{}

\newcommand{\sezione}[1]{\hline#1 & Descrizione\\\hline}

\begin{document}
	\maketitle

	\centering
	{\large Comandi bath}\\[0.01\textheight]

	\begin{tabularx}{\textwidth}{lX}
	\sezione{V.Ambiente}
	PATH & modificabile, sequenza di percorsi assoluti, divisi da ":", di directory contenenti eseguibili (lanciabili senza digitare path). Ricerca secondo ordine specificato in PATH, si ferma a primo eseguibile con nome uguale. Eventuale ErrorNotFoutd.\\

	\sezione{Directory}
	cd \textit{percorso} & Sposta logicamente in una diversa directory, secondo un path asssoluto o relativo.\\
	ls & Visualizza i files/direcotry contenuti nella directory corrente.\\
	ls -a & ls ma mosta anche file nascosti (anche ., ..).\\
	ls -l & ls ma mostra più informazioni sui files.\\
	pwd & Visualizza il percorso assoluto, da / fino alla directory corrente.\\
	touch \textit{nomeFile.estensione} & Crea un file vuoto nella dyrectory corrente.\\

	\sezione{Scripting}
	echo \textit{testo} & Visualizza a video la sequenza di caratteri passata fino al primo "INVIO".\\
	\ & Disabila interpretazione per il carattere successivo, andata a capo, permettendo di stamparlo.\\
	echo "\textit{testo}" & Come echo, ma disabilita interpretazione caratteri speciali e andate a capo.\\
	; & Separatore di più comandi lanciati durante la stessa digitazione.\\
	nome=\textit{valore} & simboli con nome e valore, stringa modificabile, alfanumerici casesensitive. No spazi prima o dopo "=". Sono d'ambiente o ex-novo.

	Solo la bash in cui sono create le variabili le può usare. I programmi lanciati dalla bash hanno una speudocopia della bash.\\
	\$\{nomeVari\} & Fa sostituire il nome della variabile con il valore. graffe opzionali se nome variabile seguito da uno spazio.\\
	\#... & Commeto.\\
	\#!... & Se indicato nella prima riga indica quale interprete deve eseguire lo script. Se non specificato usato quello corrente.\\
	\textit{comando1} | \textit{comando2} & pipe (speudo-file temporaneo): collega automaticamente l'output di un comando all'input di un altro. Unidirezionale.\\

	\sezione{Privilegi}
	chmod u+x \textit{script.sh} & Modifica permessi file mediante formato numerico: u+x terna 0-7. Ogni numero è la somma dei valori associati hai permessi di r(4), w(2), x/s(1). Ordine: proprietario, gruppo, altri utenti.\\
	chgrp ??? & Modifica il gruppo di appartenenza di un file.\\
	chown \textit{newOwner nameFile} & Modifica il proprietario (e anche gruppo) di un file.\\
	ls -al \textit{nomeFile.estensione} & ls ma mostra permessi, anche dei file nascosti. Interpretazione: 1°carattere tipo file (- file, d dyrectory, c collegamento seriale, b device a blocchi), 9 caratteri successivi terzine di permessi (r read,w write,x/s execute) per proprietario, gruppo, altri utenti. \\

	\sezione{Vari}
	bash & Crea una bash figlia, copiaparziale della padre.\\
	exit & Termina bash corrente e sale alla padre.\\
	ps & Mostra i processi in esecuzione. Dati: PID=identificatore univoco, PPID=PID processo padre .\\
	strace \textit{comando} & Dice la sistem-call usata.\\
	clear & Pulisce la CLI. Non modifica variabili create.\\
	which \textit{comando} & Cerca in PATH il comando e se lo trova mostra il path in cui si trova. \\
	cat \textit{nomeFile.estensione} & Visualizza il contenuto del file.\\
	lsmod & Elenca tutti i moduli attivi.\\
	modinfo \textit{nomeModulo} & Dice le informazioni sul modulo specificato.\\
	sudo modprobe \textit{nomeModulo} & Carica il modulo specificato.\\
	\hline
	\end{tabularx}

\end{document}