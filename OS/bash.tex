\section{Bash}

\tabella{Variabili Ambiente}{
	\riga{PATH}{Modificabile, sequenza di percorsi assoluti, divisi da ":", di directory contenenti eseguibili (lanciabili senza digitare path). Ricerca secondo ordine specificato in PATH, si ferma a primo eseguibile con nome uguale. Eventuale ErrorNotFoutd.

	Altre variabili d'ambiente: \$HOME, \$USER, \$SHELL, \$TERM.}
	\riga{\$?}{Modificato alla terminazione di ogni script, contiene l'\textit{exit status}.}
	\riga{env}{Visualizza l'elenco delle variabili d'ambiente.}
}

\tabella{File speciali}{
	\riga{/etc/passwd}{Righe sono info ogni utente divise da “:”.}
	\riga{/etc/shadow}{Righe sono password utente codificate.}
	\riga{/etc/group}{Righe sono info ongi gruppo divise da ":".}
	\riga{/usr/bin/passwd}{Cambia la pass utente.}
}

\tabella{Directory}{
	\riga{cd \textit{percorso}}{Sposta logicamente in una diversa directory, secondo un path asssoluto o relativo.}
	\riga{mkdir \textit{nomeDir}}{Crea una nuova directory.}
	\riga{touch \textit{nomeFile.estensione}}{Crea un file vuoto nella dyrectory corrente.}
	\riga{rmdir \textit{nomeDir}}{Rimuove una directory solo se è vuota.}
	\riga{rm \textit{file\textbar dir}}{Rimuove una directory vuota o un file. Parametri:
	\begin{itemize}
		\item \textbf{-r} elimina ricorsivamente sotto cartelle e file.
		\item \textbf{-f} non fa chiedere le autorizzazioni di eliminazione.
	\end{itemize}}
	\riga{mv file1 file2\textbar dir}{Rinomina file1 in file2 o sposta file1 nella directory specificata.}
	\riga{cp file1 dir}{Copia file1 nella directory specificata.}
	\riga{ls [\textit{nomefile}]}{Visualizza i files/direcotry contenuti nella directory corrente. Parametri:
	\begin{itemize}
		\item \textbf{-a} mosta anche file nascosti (anche ., ..).
		\item \textbf{-l} mostra più informazioni sui files.
		\item \textbf{-h} rende i dati più leggibili.
		\item \textbf{-d} fa applicare il comando alla directory e non ai file.
		\item \textbf{-R} mostra ricorsivamente contenuto sotto directory.
	\end{itemize}
	Se specifico un file mi dice se esiste e mi da informazioni solo di lui.}

	\riga{pwd}{Visualizza il percorso assoluto, da / fino alla directory corrente.}
}

\tabella{Controllo Comandi}{
	\riga{\textbackslash}{Disabila interpretazione per il carattere successivo, andata a capo, permettendo di stamparlo.}
	\riga{"..."}{Delimita un argomento e non fa interpretare nessun comando a eccezzione dell'espansione di variabili (\$..) e l'esecuzione di comandi.}
	\riga{'...'}{Delimita un argomento e non fa interpretare nessun comando.}

	\riga{pre\{s1,...\}post}{Stringa di testo racchiusa fra separatori (spazio, tab, a capo) con coppia di graffe (non precedute da \$) e senza separatori.

		Le stringhe racchiuse dalle graffe vengono composte con il preambolo (pre) e postscritto (post), che sono opzionali. Alternative:

		Sono annidabili (quelle più esterne eseguite per prime).
		Vengono eseguite prima le brace expansions delle variable expansions.

	\begin{itemize}
			\item $a_1..a_2$ lettere da $a_1$ a $a_2$ nell'alfabeto.
			\item $n_1..n_2$ numeri compresi tra $n_1$ e $n_2$.
	\end{itemize}}
	\riga{\textit{cmd1} ; ...}{Separatore di più comandi, e dei rispettivi argomenti, scritti sulla stessa riga di comando e eseguiti dopo la terminazione del precedente (lista di comandi).

	L'exit status è quello dell'ultimo comando lanciato.

	Se racciusi da () eseguiti in una sub-schell.}
	\riga{\textit{cmd1} \textbar\textbar \textit{cmd2}}{Esegue cmd1 e solo se cmd1 fallisce (exit status$\ne$0) esegue cmd2.}
	\riga{\textit{cmd1} \&\& \textit{cmd2}}{Esegue cmd1 e solo se cmd1 ha successo (exit status=0) esegue cmd2.}
	\riga{[[...]]}{Espressione condizionale, usa \&\& e \textbar\textbar, che restituisce 0=true, altro=false.}
	\riga{*}{Sostituito con una qualsiasi sequenza di caratteri anche vuota.}
	\riga{?}{Sostituito con un singolo carattere (no spazio vuoto).}
	\riga{[c1c2...]}{Sostituito con solo uno dei caratteri specificati in elenco. Alternative:
	\begin{itemize}
		\item $a_1..a_2$ lettere da $a_1$ a $a_2$ nell'alfabeto.
		\item $n_1..n_2$ numeri compresi tra $n_1$ e $n_2$.
		\item \textbf{[:digit:]} una cifra.
		\item \textbf{[:upper:]} un carattere maiuscolo.
		\item \textbf{[:lower:]} un carattere minuscolo.
	\end{itemize}
	Annidabili.}
}

\tabella{Scripting}{
	\riga{echo \textit{testo}}{Visualizza a video la sequenza di caratteri passata fino al primo "INVIO". Se passo \textit{"testo"} si disabilità l'interpretazione dei caratteri speciali e andate a capo.}

	\riga{nome=\textit{valore}}{Simboli con nome e valore, stringa modificabile, alfanumerici casesensitive. No spazi prima o dopo "=". Sono d'ambiente o ex-novo locali.

	Solo la bash in cui sono create le variabili le può usare. I programmi lanciati dalla bash hanno una speudocopia della bash.}

	\riga{\$variabile}{Fa l'espanzione della variabile, ovvero la stostituisce con il suo contenuto.}
	\riga{\$\{!variabile\}}{Fa l'espanzione della variabile che contiene il nome d'unaltra variabile con il valore di quest'ultima (riferimento indiretto).

	Dalla versione 2 di bash.}

	\riga{export nomevar \textbar nomevar=valore}{Variabile d'ambiente. Un shell figlia riceve una copia modificabile che non influenza variabile d'ambiente del padre.}
	\riga{unset \textit{nomevariabile}}{Elimina una variabile esistente (vuota o no).

	Quotare ("...") sempre variabili per evitare errori con variabili vuote o inesistenti.}
	\riga{\$\{nomeVar\}}{Fa sostituire il nome della variabile con il valore. graffe opzionali se nome variabile seguito da uno spazio.}
	\riga{\#...}{Commeto.}
	\riga{\#!...}{Se indicato nella prima riga indica quale interprete deve eseguire lo script. Se non specificato usato quello corrente.}
	\riga{\textit{comando1} \textbar \textit{comando2}}{pipe (speudo-file temporaneo): collega automaticamente l'output di un comando all'input di un altro. Unidirezionale sinDes.}
	\riga{\textit{script.sh} c1...}{Sono un insieme ordinato di caratteri separati da spazi successivi al nome del programma. Sono immodificabili dopo la sostituzione dei metacaratteri (*, ?,ecc).

	Riga di comando = nomeProgramma + parametri.
	Nella riga di comando gli elementi sono indicizzati da 0 (nomeProgramma).
	\begin{itemize}
		\item \textbf{\$\#} Contiene il numero di parametri passati.
		\item \textbf{\$n} Accede all'n-esimo parametro a partire da indice 0.
		\item \textbf{\$*} Tutti argomenti concatenati e divisi da spazi.
		\item \textbf{\$@} Vettore di argomenti quotati ("...").
	\end{itemize}
	I parametri \$* e \$@ sono identici se non quotati (concatenazione di argomenti separati da " "). Se quotati \$* quota tutti gli argomenti assieme mentre \$@ quota singolarmente ogni argomento.

	\$@ è usato per passare parametri a comandi dentro a degli script.}
	\riga{`./script.sh`}{Sostituisce a run-time il comando con l'output (stdout) dello script specificato. Comando alternativo \$(./script.sh)}
}

\tabella{Espressioni aritmetica}{
	\riga{((...))}{Valuta una stringa come un espressione aritmetica (+,-,*,/,\%, ()) di soli interi. Racchiude un espressione più eventualmente un assegnamento. Si possono usare variabili nell'espressione (\$variabile).

	L'exit statuts 0 indica espressione true, diverso da 0 indica espressione false.

	Per le operazioni in virgola mobile usare \textbf{bc}.}
	\riga{\$((...))}{Come operatore ((...)) ma è concatenabile con stringhe tramite " ".}
}

\tabella{Tilde Expansion}{
	\riga{~/...}{Tilde espansa con il percorso assoluto della home directory dell'effective user. Valido caso con solo \textbf{~} e solo \textbf{~/}.}
	\riga{~userName/...}{Tilde e userName espansi con il percorso assoluto della home directory dell'utente specificato. Valido caso con solo \textbf{~userName/}.}
}

\tabella{Privilegi}{
	\riga{chmod u+x \textit{script.sh}}{Modifica permessi file mediante formato numerico: u+x terna 0-7. Ogni numero è la somma dei valori associati hai permessi di r(4), w(2), x/s(1). Ordine: proprietario, gruppo, altri utenti.

	Può diventare un quartetto aggiungendo per primo l'identificatore numero dei privilegi di \textit{setuid, setgid, sticky bit}.}
	\riga{chgrp ???}{Modifica il gruppo di appartenenza di un file.}
	\riga{chown \textit{newOwner nameFile}}{Modifica il proprietario (e anche gruppo) di un file.}
	\riga{ls -al \textit{nomeFile.estensione}}{Mostra permessi, anche dei file nascosti. Interpretazione: 1°carattere tipo file (- file, d dyrectory, c collegamento seriale, b device a blocchi), 9 caratteri successivi terzine di permessi (r read,w write,x/s execute) per proprietario, gruppo, altri utenti.}
	\riga{whoami}{Dice all'utente corrente le sue informazioni.}
	\riga{sudo \textit{comando}}{fa eseguire il comando come administratore, può essere chiesta userPass. Solo utenti gruppo sudo (gestito dall'admin) possono usarlo.}
}


\tabella{Subshell}{
	\riga{bash}{Crea una shell figlia (interattiva non di login).

	Eredita dal padre: dir. corrente, copia variabili d'ambiente.  Non sono ereditate le variabili locali.

	Creata in automatico per comandi raggruppati, script, processi in background. I comandi built-in sono  eseguiti in shell corrente/padre.
	\begin{itemize}
		\item \textbf{-c \textit{script.sh}} non interattiva.
		\item \textbf{-l \textbar --login} interattiva di login.
	\end{itemize}}

	\riga{var=val comando}{Scrivendo le assegnazioni prima dell'esecuzione di un comando si creano delle var. d'ambiente solo per l'imminente subshell. Non saranno ereditate da successive subshell.}
	\riga{. \textbar source \textit{script.sh}}{Esegue lo script nella shell corrente.

	Utile a impostare/modificare variabili shell.

	Ignorata prima riga opzionale e eseguito con interprete corrente.}
	\riga{exit}{Termina bash corrente, elimina l'ambiente e sale alla padre.}
	\riga{exit \textit{exitStatus}}{Termina lo script restituendo un valore intero [0-255] per indicarne l'esito di terminazione. 0 indica esecuzione terminata senza errori, qualcosaltro indica un'errore. Viene restituito alla shell esecutrice. }
	\riga{top}{Mostra in tempo reale processi in esecuzione e risorse di sistema usate.}
	\riga{ps}{Mostra i processi in esecuzione.

	Con l'opzione \textit{-all} vedo più informazioni (PID, PPID,ecc).}
	\riga{set}{Visualizza sia variabili locali che d'ambiente della shell corrente (anche funzioni di shell). I parametri [re]settano dei comportamenti della shell:
	\begin{itemize}
		\item \textbf{+o \textit{comando}} Disalibilita il comando (tipo history).
		\item \textbf{-o \textit{comando}} Abilita il comando (tipo history).
		\item \textbf{-a} Successive variabili create/modificate diverrano d'ambiente e ereditate da shell figlie. Per [ri]definire variabili locali usare \textbf{export -n \textit{variabile}}.
		\item \textbf{+a} Successive variabili create/modificate diverrano locali e non ereditabili da shell figlie. (default).
	\end{itemize}}
}

\tabella{Vari}{
	\riga{./ \textit{conado}}{Esegue comando presente nella directory corrente (percorso relativo).

	Alternativa è inserire il percorso relativo o aggiungere il percorso assoluto alla variabile PATH.}
	\riga{strace \textit{comando}}{Dice la sistem-call usata.}
	\riga{clear}{Pulisce la CLI. Non modifica variabili create.}
	\riga{which \textit{comando}}{Cerca in PATH il comando e se lo trova mostra il path in cui si trova.}
	\riga{cat \textit{nomeFile.estensione}}{Visualizza il contenuto del file.}
	\riga{lsmod}{Elenca tutti i moduli attivi.}
	\riga{modinfo \textit{nomeModulo}}{Dice le informazioni sul modulo specificato.}
	\riga{sudo modprobe \textit{nomeModulo}}{Carica il modulo specificato.}
	\riga{history}{Visualizza comandi, numerati, precedentemente eseguiti, anche da shell precedentemente chiuse.

	Con !NUMERO lancio il comando corrispondente nell'elenco di history. Con !STRINGA lancio il comando più recente che corrisponde alla stringa.}
	\riga{greap \textit{stringa}}{Seleziona le righe contenenti la stringa passata (che sarà evidenziata) dal testo che sarà successivamente scritto. Lo si interrompe con "Ctrol+d".

	Spesso usato come secondo comando di \textbar per filtrare output di altri comandi.}
	\riga{wc}{Dice il testo passatogli il numero di righe, parole e caratteri.}
}

Directory "proc" info sui processi memorizzati con una dir. per ogni processo attivo, create e cancellate continuamente.
Directory sulla ram "/prc/ sono i processi in esecuzione e nella directory "fd" ci sono i file descriptor dei file aperti.

\tabella{Costrutti controllo flusso}{
	\riga{for varName in elencoParole ; do listCommand ; done}{Dopo il for variabile è usabile.}
	\riga{for ((exp1 ; exp2 ; exp3)) ; do listCommand ; done}{Le exp sono espressioni valutate aritmeticamente. }
	\riga{if listA ; then listB ; [elif listC ; then listD ; ...] ... [else listZ ;]}{...}
	\riga{while list ; do list ; done}{...}
}

In tutti i costrutti l'exit value è o quello della lista do comandi o 0 se nessun comando viene eseguito.

Espressioni condizionate su file o variabile: [ condizione di un file ].
Valutazione di un espressione matematica applicata a variabili d'ambiente: (( istruzini con espressione ))