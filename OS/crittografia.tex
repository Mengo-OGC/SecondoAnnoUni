\section{Crittografia}
Algoritmo pubblico/noto, robusto e chiavi segrete (o note in parte) di varie dimensioni.
\begin{itemize}
	\item Codificare (E): $testoOriginale + chiaveCodifica \stackrel{codifica}{\longrightarrow} testoCodificato$.
	\item Decodificare (D): $testoCodificato + chiaveDecodifica \stackrel{decodifica}{\longrightarrow} testoOriginale$.
\end{itemize}

Impossibile ottenere testo orignale dal testo cifrato senza conoscere la chiave.

\subsection{Crittografia Simmetrica (chiave segreta)}
Stessa chiave $K$ per codificare e decodificare per $A$ e $B$.

+Veloce, +Efficente, -distribuzione chiavi fase risciosa, -tante chiavi (n*(n-1)/2).

\begin{center}
	\begin{tikzcd}
		A_K \arrow[r, "testo"] & E_K \arrow[r, "E_K(testo)"] & D_K \arrow[r, "testo"] & B_K
	\end{tikzcd}
\end{center}

Algoritmo noto a tutti, chiave nota solo a $A$ e $B$.
\begin{itemize}
	\item SI: segretezza, autenticazione, integrità
	\item NO: Non repudiabilità
\end{itemize}

Adottato da Des, Triple-DES, AES, ecc.

\subsection{Crittograffia Asimmetrica (chiave pubblica)}
Una chiave pubblica/nota ($K^+$) e una chave privata/segreta ($K^-$) sia per $A$ che per $B$ \textbf{correlate}\footnote{Un testo cifrato con una chiave può essere decrifrato solo dalla chiave a cui è correlata (legata).}.
$A$ e $B$ non codividono chiavi segrete.

Due chiavi private-pubbliche (correlate) criptano e decriptano in modo reciproco. La chiave privata è incalcolabile dalla chiave pubblica.

+Sicurezza (non scambio completo chiavi), -Lento, -azzenza chiavi pubbliche proprietà.

Configurazioni:
\begin{itemize}
	\item $A$ sa a chi spedisce ma $B$ non sa chi gli scrive.
		\begin{center}
			\begin{tikzcd}
				A_{K^-_A}^{K^+_A} \arrow[r, "testo"] & E (K_B^+) \arrow[rr, "K_B^+(testo)"] &  & D (K_B^-) \arrow[r, "testo"] & B_{K^-_B}^{K^+_B}
			\end{tikzcd}
		\end{center}

		SI: segretezza.

	\item $A$ non sa chi capisce cosa scrive ma $B$ sa chi scrive.
		\begin{center}
			\begin{tikzcd}
				A_{K^-_A}^{K^+_A} \arrow[r, "testo"] & E (K_A^-) \arrow[rr, "K_A^-(testo)"] &  & D (K_A^+) \arrow[r, "testo"] & B_{K^-_B}^{K^+_B}
			\end{tikzcd}
		\end{center}

	SI: Autenticazione, Integrità, Non repudiabilità.

	\item $A$ sa chi capisce cosa scrive e $B$ sa chi scrive.

	\begin{center}
		\begin{tikzcd}
			A_{K^-_A}^{K^+_A} \arrow[d, "testo"']          &  & B_{K^-_B}^{K^+_B}                      \\
			E_1 (K_A^-) \arrow[d, "K_A^-(testo)"']         &  & D_2(K_A^+) \arrow[u, "testo"']         \\
			E_2 (K_B^+) \arrow[rr, "K_B^+(K_A^-(testo))"'] &  & D_1 (K_B^-) \arrow[u, "K_A^-(testo)"']
		\end{tikzcd}
	\end{center}

	SI: Segretezza, Autenticazione, Integrità, Non repudiabilità.
\end{itemize}

Adottato da RSA, ecc.

\subsubsection{Funzioni Hash Crittografiche}
Una \textbf{Funzione Hash crittografiche ($MD()$)} (one-way hash, funzioni digest) prende in input un messaggio (m) lungo a piacere e da in output un digest (stringa) a lunghezza fissa ($MD(m)$).

Il digest non è reversibile, ossia non esise $MD(x)=MD(y)$ oppure dato $y=MD(x), x$ è indeterminabile.

Più due $m$ sono simili più i digest sono diversi. Più due $m$ sono diversi non è detto che i digest siano molto diversi.

Le hash non sono decodificabili, non servono chiavi di cifratura. Le funzioni di codifica/decodifica non restituiscono output a lunghezza fissa.

Hash è diverso per quanto simili dal checksum (perchè è reversibile).

Funzioni hash: MD5, SHA-1/2(x, 256, 512, ecc)/3, ecc.

\subsection{MAC e Firma digitale}
MAC (Message Authentication Code) è una tecnica, veloce, di autenticazione parziale dell'origine di un messaggio. Inoltre garantisce l'integrità del messaggio.

\bigskip

\textbf{MAC}: Si attua combinando crittografia simmetrica e funzioni Hash crittografico. Attori devono conoscere la chiave segreta. Assente autenticazione di chi spedisce (potrebbe essere l'altro o sestesso). SI: Integrità, autenticazione.

\textbf{Firma Digitale}: Tecnica di MAC avanzata. Si attua combinando crittografia asimmetrica e funzioni Hash crittografico. Mit. deve avere chiavi assimetriche. Chiunque può leggere e verificare l'integrità. SI: Integrità, Autenticazione, Non repudiabilità.

\textbf{Spedizione}:
\begin{enumerate}
	\item Applico Funzione hash crittografico a m ottenendo MD(m).
	\item Codifico il diget simmetricamente o asimmetricamente (con la chiave privata del mittente) ottenendo il MAC.
	\item Concatenazione MAC al m.
	\item Spedizione (m, MAC) che può modificare m in (m', MAC).
\end{enumerate}

\textbf{Ricezione}:
\begin{enumerate}
	\item Separazione di (m', MAC).
	\item Decodifica simmetrica o asimmetrica (usa chiave pubblica mittente) del MAC ottenendo il digest (MD(m)).

	Terminazione improvvisa se decodifica non da successio. Implica che mittente non è quello atteso.
	\item Applico Funzione hash crittografico a m' ottenendo MD(m').
	\item Confronto i due digest. Se sono uguali m' è l'originale spedito dal mittente dichiarato.
\end{enumerate}

\begin{center}
	\begin{adjustbox}{max width=\textwidth}
		\begin{tikzcd}
			m \arrow[rrrrr, bend left] \arrow[rr, "hash"] &          & MD(m) \arrow[rr, "codifica"]            &                  & MAC \arrow[r]                & {(m, MAC)} \arrow[d, "Spedizione"'] \\
			errori                                        & successo & uguali? \arrow[ll, bend left] \arrow[l] & MD(m') \arrow[l] & m' \arrow[l, "hash"']        & {(m',MAC)} \arrow[ld] \arrow[l]     \\
			&          &                                         & MD(m) \arrow[lu] & MAC \arrow[l, "decodifica"'] &
		\end{tikzcd}
	\end{adjustbox}
\end{center}

\subsection{GPG e PGP}
GPG meccanismo di firma digitale gratuito basato su PGP proprietario. Garantisce autenticità e integrità di un file/package.

\bigskip

Il \textbf{maintainer}\footnote{Qualcuno che distribuisce, assicura il funzionamento e aggiornamento di una risorsa online.} della risorsa possiede chiavi asimmetriche e per ogni risorsa da pubblicare crea la firma digitale e la distribuisce  con la risorsa.
Oggi firmato solo file indice di risorse  (Release.gpg) che ha versione e checksum (digest) di ogni risorsa.

L'utente, scarica se non l'ha la chiave pubblica, scarica il package+firma cerifica firma con la chiave pubblica del mainteiner.

Le chiavi pubbliche si possono trovare di default  nell'OS o si scaricano da canali ufficiali verificando a modo.