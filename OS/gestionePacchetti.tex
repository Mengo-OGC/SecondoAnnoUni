\section{Gestione Pacchetti in Linux}

\subsection{Concetti chiave}
\begin{itemize}
	\item \textbf{Package (.deb)}: archivio compresso di file descrivono programma e metadati del package.
	\item \textbf{Dipendenze}: Librerie necessarie funzionamento programma.
	\item \textbf{Repository}: Collezioni locali o remote di package in server centralizzati.
	\item \textbf{Database locale di package}: Elenco locale dei metadati di package nei repository.
\end{itemize}

\subsection{Installazione Package}
Interfaccia a CLI per download-installazione (risolve dipendenze e gestisce fonti), rimozione, aggiornamento, gestione di package.
Usa \texttt{dpkg} per l'installazione vera e propria (disassembla package e installa in directory giuste) e \texttt{apt-cache} per la ricerca (usabili anche dall'utente).

Assente di descrizione esecuzioni (ideale per script).

\medskip
\texttt{apt-get}
\medskip

\texttt{apt} fornisce una descrione testuale della procedura eseguita (user freandly).

\begin{center}
	\begin{tabularx}{\textwidth}{|lX|}
		\hline
		\rowcolor{gray!20}
		\textbf{Comandi apt-get} & \textbf{Descrizione}\\
		\hline
		update & Aggiorna database locale di package. Da eseguire per primo.\\
		upgade & Installa le nuove versioni dei package già installati.\\
		install & Installa un package nuovo e le dipendenze.\\
		remove & Rimuove package lasciando file configurazione.\\
		purge & Rimuove package e file configurazione.\\
		autoremove & Rimuove package, non più necessari, autoinstallati soddisfare dipendenze altri package.\\
		dist-upgrade & Aggiorna la distribuzione (package e dipendenze).\\
		clean & Cancella package dalla cache locale.\\
		\hline
	\end{tabularx}
\end{center}
Formato generico: \texttt{sudo apt-get "comando" "packageOpzionale"}

Procedura Installazione Package:
\begin{enumerate}
	\item \texttt{sudo apt-get update} Aggiorna l'indice/database locale scaricando l'elenco aggiornato dei package e versioni dai repository.
	\item \texttt{apt-get intall "package"} Installa il package cercandolo nel database locale e controllando le dipendenze.
\end{enumerate}

\subsubsection{Posizione Repository}
I database locali di package sono contenuti in: \texttt{/etc/apt/sources.list} (principale), \texttt{/etc/apt/sources.list.d/} (aggiunte repository individuali).

\bigskip

\textbf{Formato one-line}: \texttt{deb [opzioni] URI distribuzione comp1 comp2 ...}
\begin{itemize}
	\item \textbf{Tipo}: Indica il tipo d'archivio.

	\texttt{deb} indica package binari precompilati.

	\texttt{deb-src} indica package di sorgenti.

	\item \textbf{Opzioni}: La firma GPG si attua col campo dedicato che accettà il \textbf{keyring} con la public-key del mainteiner.

	\texttt{signed-by="..."}
	\item \textbf{URI}: Indirizzo del repository.
	\item \textbf{Distribuzione}: Nome in codice della versione del OS.
	\item \textbf{Componenti}: Definiscono il tipo di licenza (main, restricted, universe, multiverse).
\end{itemize}

Caratteristiche: chiarezza, leggibilità, facile da gestire e sicurezza esplicità.

C'è anche un formato deb822 più leggibile.
\bigskip

Directory public-key: \texttt{/usr/share/keyrings/} (key di sistema), \texttt{/etc/apt/keyrings/} (key d'utente).

\bigskip

Il \texttt{keyrings, sources.list, sources.list.d/} sono editabili con editor con privilegio sudo o GUI specifiche.

Aggiunta Manuale:
\begin{enumerate}
	\item Verificare esisetnza directory, scarica GPG-key (sopprime output e reindirizza allo stdout), comprimerla (ASCII in binario .gpg) e salvataggio nel keyring.

	\texttt{sudo mkdir -p /etc/apt/keyrings}

	\texttt{wget -qO- URI | sudo gpg --dearmor -o directory.gpg}

	\item Aggiungere riferimento one-line con signed-by del repository alla directory.

	\texttt{echo "FormatoOneLine" | sudo tee directory.list}

	\item Aggiornamento e installazione: \texttt{sudo apt update}, \texttt{sudo apt install nomePackage}.
\end{enumerate}



