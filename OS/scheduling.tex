\section{Esercizi Scheduling}
Procedura risolutiva di un esercizio di scheduling data la \textbf{durata del quanto di tempo} e una tabella dei processi del tipo:

\rowcolors{2}{white}{white}

\begin{center}
	\begin{tabular}{lll}
		Processi & Tempo di Arrivo & Tempo d'Esecuzione \\
		\hline
		... & ... & ...
	\end{tabular}
\end{center}

\textbf{*Passo 1*}

Fare una  tabella di scheduling come la seguente:

\begin{center}
	\begin{tabular}{l|llll}
		Ready Queue & & & & \\
		Running & & & & \\
		Quanto d'Inizio/Fine & & & & \\
		\hline
		Tempo & 0 & 1 & ... & n \\
		\hline
		T.R. P1 & & & & \\
		... & & & & \\
		T.R. PN & & & & \\
	\end{tabular}
\end{center}

Spiegazione delle righe della tabella:
\begin{enumerate}
	\item \textit{Ready Queue} specifica per ogni quando i processi pronti per esere eseguiti, tipicamente è FIFO.
	\item \textit{Running} indica per ogni quanto di tempo il processo in esecuzione.
	\item \textit{Quanto d'Inizio/Fine} evidenzia i quanti d'arrivo (A) e di termine (F) dei processi.
	\item \textit{Tempo} sono i quanti del processore. Ogni colonna è un quanto.
	\item \textit{Righe T.R.} tempo d'esecuzione rimante per ogni processo a ogni quanto di tempo dalla sua partenza.
\end{enumerate}

\textbf{*Passo 2*}

Per rispondere alle statistiche richieste fare una tabella come la seguente:

\begin{center}
	\begin{tabular}{llll}
	Processi & T. Turnaround & Intervalli di Ready & T. Attesa \\
	\hline
	...	& ... & [x1-x2], ... & ... \\
	\hline
	Medie: & ... & ... & ... \\\
	\end{tabular}
\end{center}

Spiegazione delle colonne della tabella:
\begin{enumerate}
	\item \textit{Turnaround} = T. fine - T. arrivo.
	\item \textit{Intervalli di Ready} sono i quanti in cui il processo è stato nella coda di ready (dal quanto in cui vi è entrato al quanto in cui vi è uscito).
	\item \textit{T. Attesa} somma degli intervalli di ready.
\end{enumerate}